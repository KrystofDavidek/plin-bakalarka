%!TEX program = xelatex
\documentclass[a4paper,12pt,openany,twoside]{book} % twoside
\usepackage[inner=3.5cm,outer=2.5cm, top=2.5cm]{geometry} %showframe

\usepackage{url}
\usepackage{fontspec}
\usepackage{lmodern}
\usepackage{csquotes}
\usepackage{graphicx}
\usepackage{caption}
\usepackage{subcaption}
\usepackage[rgb]{xcolor}
\graphicspath{ {assets/} }
\usepackage{xunicode}
\usepackage{xltxtra}
\usepackage{chngcntr}
\usepackage[czech]{babel}
\usepackage[pagestyles,medium]{titlesec}
\usepackage{setspace}
\usepackage{emptypage}
\usepackage{xpatch}
\usepackage{caption}
\usepackage{subcaption}
\usepackage{ragged2e}
\usepackage{pbox}
\usepackage{rotating}
\usepackage{float}
\usepackage{afterpage}
\usepackage{tipa}

\usepackage[hidelinks]{hyperref}

\usepackage{mdframed}
\usepackage{framed}
\def\tightlist{}

\xpatchcmd{\part}{\thispagestyle{plain}}{\thispagestyle{empty}}{}{}


\newcommand\exmp{\textsf}

\counterwithout{figure}{chapter}
\counterwithout{footnote}{chapter}


\newpagestyle{sensible}{
	\headrule\sethead{}{}{\MakeUppercase{\chaptertitle}}
	\setfoot{}{\thepage}{}
}

\setlength\emergencystretch{1em}
\setlength\headheight{14pt}
\setstretch{1.1}
\setcounter{secnumdepth}{4}

\widowpenalty10000
\clubpenalty10000

\hyphenation{выпа-дений}

\setmainfont[Ligatures=TeX]{CMU Serif Roman}

\usepackage[backend=biber, style=iso-authoryear, sortlocale=cs\_CZ, uniquelist=false, autocite=footnote, maxcitenames=2, maxbibnames=99, minnames=1, urldate=long, spacecolon=false,bibencoding=UTF8
]{biblatex}
\let\oldmultinamedelim\multinamedelim
\let\oldfinalnamedelim\finalnamedelim
\renewcommand*{\multinamedelim}{~a~}
\renewcommand*{\finalnamedelim}{~a~}
\renewcommand*{\nameyeardelim}{~}
\AtBeginBibliography{%
  \renewcommand*{\multinamedelim}{~--\space}%
  \renewcommand*{\finalnamedelim}{~--\space}%
}
\DefineBibliographyStrings{czech}{%
  mathesis = {Bakalářská diplomová práce},
  editors = {eds.}
}

\addbibresource{bibliography.bib}

\newcommand{\sign}[1]{%      
  \begin{tabular}[t]{@{}r@{}}
  \makebox[2.5in]{\dotfill}\\
  \strut#1\strut
  \end{tabular}%
}


\begin{document}
	\clearpage
	\pagenumbering{gobble}

	\begin{titlepage}
		\begin{center}
			{\Large\uppercase{Masarykova univerzita}}

			\vspace{1em}

			{\Large Filozofická fakulta}

			\vspace{1em}

			{\large Ústav českého jazyka}

			\vspace{1em}

			{\large Český jazyk se specializací počítačová lingvistika}

			\vspace{11em}

			{\large Kryštof Davídek }
			
			\vspace{3em}
			
			{\LARGE\bf Elektronický slovník s definicemi založenými na derivačních rysech slovotvorně motivovaných slov}

			\vspace{1.5em}

			{\Large Bakalářská diplomová práce}

			\vfill
			\vspace{3em}
			Vedoucí práce: Mgr. Adriana Válková
			
			2019
		\end{center}
	\end{titlepage}


\cleardoublepage

\par
\par\vspace*{\fill}
	\pagestyle{plain}
\pagenumbering{roman}
\begin{flushright}
	Prohlašuji, že jsem bakalářskou diplomovou práci vypracoval samostatně s~využitím uvedených pramenů a~literatury.

	\vspace{3em}

	    \makebox[2.5in][r]{\dotfill}
	    
	    Kryštof Davídek

	    \par

\end{flushright}
\clearpage

\par
\par\vspace*{\fill}

Na tomto místě bych rád poděkoval...

\clearpage

\section*{Abstrakt}

Tato bakalářská práce se zabývá aplikační rovinou slovotvorné derivace. Cílem práce je navrhnout a implementovat elektronický slovník s definicemi založenými na derivačních rysech slovotvorně motivovaných slov ve formě mobilní aplikace. Pro vytváření slovotvorných definic byla využita volně přístupná data z derivační sítě DeriNet. Předkládaný nástroj byl vyvíjen ve frameworku Ionic a praktickým výstupem této práce je mobilní aplikace pro operační systém Android. Primárním účelem této aplikace je podporovat výuku češtiny jako druhého jazyka z hlediska znalosti slovotvorných prostředků.

\section*{Abstract}

This bachelor thesis is focusing on the application level of word derivation. The purpose of the thesis is to design and implement an electronic dictionary with definitions for motivated words based on derivation features in the form of a mobile application. For creating word-forming definitions, freely accessible data of the derivation network DeriNet was used. This tool was developed in the Ionic framework and the practical output of this work is a mobile application for the Android operating system. The primary purpose of this application is to support the teaching of Czech as a second language in terms of knowledge of word-forming resources.

\section*{Klíčová slova}

slovotvorba, derivace, derivační morfologie, mobilní aplikace, DeriNet, Ionic, Angular

\section*{Keywords}

word formation, derivation, derivational morpohology, mobile applications, DeriNet, Ionic, Angular

\clearpage

\tableofcontents


\cleardoublepage
\pagenumbering{gobble}

% \newgeometry{top=2.5cm}

\pagenumbering{arabic}
\hypertarget{uxfavod}{%
\chapter*{Úvod}\label{uvod}
\addcontentsline{toc}{chapter}{Úvod}}

Odvozování slov je v~českém jazyce považováno \ldots{}

\hypertarget{slovotvorba}{%
\chapter{Slovotvorba}\label{slovotvorba}}

V~této úvodní kapitole si představíme slovotvorbu tak, jak ji vnímá
dnešní tuzemská lingvistika, a~ve stručnosti si popíšeme hlavní přístupy
k~této problematice v~českém prostředí.

\hypertarget{uvedenuxed-do-problematiky}{%
\section{Uvedení do problematiky}\label{uvedenuxed-do-problematiky}}

Za slovotvorbu (derivologii) lze v~lingvistice v~nejobecnějším pojetí
považovat nauku o~tvoření slov, jde tedy o~takovou vědní disciplínu,
která zkoumá a~popisuje procesy, které doprovází vznik nových slov
v~daném jazyce.

V~užším pojetí pod tvořením slov myslíme právě ty procesy, jež pod sebou
zahrnují různé způsoby, postupy a~prostředky, díky nimž dochází
k~produkci nových slov, tedy neologismů. (Tyto slovotvorné postupy jsou
často reprodukovány.) Důležitou součástí slovotvorby jsou pak výsledné
stavy těchto procesů, které označujeme jako slovotvornou stavbu.
\parentcite[92]{cechova00}

Ze slovotvorné stavby lze typicky odvodit samotnou genezi daného
neologismu, to znamená, že by mělo být zřejmé, z~jakého slova je slovo
nové vytvořeno (například u~slova \emph{ne-dobrý} lze lehko vyčíst, že
je pomocí prefixu \emph{ne} odvozeno od adjektiva \emph{dobrý}). To ale
nemusí platit vždy, často není u~odvozovaných slov (viz kapitola NEZNÁM
JEŠTĚ ČÍSLO) úplně zřejmé, které slovo bylo tvarem základovým.
\parentcite[92--93]{cechova00}

\hypertarget{pux159uxedstupy-a-teorie}{%
\section{Přístupy a~teorie}\label{pux159uxedstupy-a-teorie}}

Jako většina lingvistických disciplín, prošla si i~tato oblast svým
vlastním vývojem, jehož dobře mapuje kapitola \emph{Slovotvorba}
v~publikaci Zdeny Rousínové -- \emph{Dějiny českého jazykovědného
myšlení}. Pro účely této práce vynecháme historický exkurz a~zaměříme se
na nejvýznamnější synchronní přístupy v~rámci tvoření slov ve 20. a~21.
století.

Za jedno z~prvních významných děl lze považovat \emph{Mluvnici spisovné
češtiny} od Františka Trávníčka (1. vydání vyšlo v~roce 1948), která
jako jedna z~prvních gramatik synchronně popisuje problematiku
slovotvorby. Zde je nutno podotknout, že zde Trávníček tvoření nových
slov úzce spojoval s~lexikologií a~nevyčlenil jej tak jako samostatnou
disciplínu.~\parencite[263]{rousinova}

Dalším výrazným příspěvkem do teorie slovotvorby přispěl Vladimír
Šmilauer svojí publikací \emph{Novočeské tvoření slov}, které bylo
napsáno v~letech 1937--1938, nicméně z~geopolitických důvodů dílo vyšlo
až v~roce 1971. Šmilauer se zde primárně zaměřil na popis procesu,
kterým si procházejí mluvčí, chtějí-li vytvářet nové pojmenování pro
nějaký jev.~\parencite[265]{rousinova}

Za přelomové dílo, které proměnilo českou slovotvorbu do dnešních podob,
je publikace Miloše Dokulila \emph{Teorie odvozování} (vyšla v~roce
1962) -- tato práce byla první částí kolektivního díla \emph{Tvoření
slov v~češtině}, na níž se podílel kolektiv pracovníků Ústavu pro jazyk
český Československé akademie věd. Cílem tohoto textu bylo podat
komplexní metodologický základ, jenž měl být svoji univerzálností platný
i~na další slovanské jazyky.~\parencite[267]{rousinova} Tato práce dala
za vznik všeobecně přijímané onomaziologické teorii slovotvorby, kterou
si důsledněji rozebereme v~druhé kapitole, nicméně je zde vhodné
poznamenat, že na Dokulilovo dílo bylo navázáno dalšími pracemi, které
jeho teorii dále rozvíjeli, aniž by jeho přístup ztratil na platnosti.
I~díky tomu se do teď tato teorie považuje za nejúplnější analýzu
slovotvorné struktury češtiny.~\parencite[272--273]{rousinova}

\hypertarget{onomaziologickuxe1-teorie-slovotvorby}{%
\chapter{Onomaziologická teorie
slovotvorby}\label{onomaziologickuxe1-teorie-slovotvorby}}

\hypertarget{zuxe1kladnuxed-charakteristika}{%
\section{Základní
charakteristika}\label{zuxe1kladnuxed-charakteristika}}

Jak bylo v~minulé kapitole naznačeno, onomaziologická teorie slovotvorby
(dále jen OTS) Miloše Dokulila výrazně ovlivnila přístup ke slovotvorbě
jako takové, protože si kladla za cíl popsat proces tvoření slov soudobé
češtiny a~zároveň definovat hranici s~dalšími lingvistickými
disciplínami jako jsou morfologie či lexikologie -- jedná se tedy o~ryze
synchronní přístup, který je založen na onomaziologii (obecné teorii
pojmenovávání).~\parencite{enc-ots17}

Onomaziologie zkoumá takové motivace, postupy a~prostředky,
prostřednictvím kterých jsou v~daném jazyce vyjadřovány určité obsahy.
Na rozdíl od sémaziologie postupuje od obsahu k~formě, to znamená, že
vychází od pojmu k~pojmenování, kterým je daný pojem vyjádřen.
\parencite{enc-onomaz17} Díky těmto termínům OTS terminologicky vyřešila
problematiku slov \emph{pojem} a~\emph{slovo}, která byla dříve často
nesystematicky zaměňována.~\parencite[267]{rousinova07}

Po metodické stránce Dokulil u~tvoření slov rozlišuje aspekt genetický,
který je zaměřený na vlastní slovotvorné procesy, a~na aspekt funkčně
strukturní, jenž sleduje výsledek těchto procesů (\emph{utvářenost
slov}). Nicméně jak sám autor teorie potvrzuje, je mezi těmito přístupy
těsná spojitost.~\parencite[9]{dokulil62}

Teorie jako taková je založena na propracovaném systému
onomaziologických a~pojmových kategoriích, které obsahy vědomí
strukturují, tedy zobecňují nebo konkretizují. Tato problematika je dále
rozebírána v~XXX.~\parencite{enc-onomaz-kateg17}

\hypertarget{zpux16fsoby-a-prostux159edky-tvoux159enuxed-novuxfdch-pojmenovuxe1nuxed}{%
\section{Způsoby a~prostředky tvoření nových
pojmenování}\label{zpux16fsoby-a-prostux159edky-tvoux159enuxed-novuxfdch-pojmenovuxe1nuxed}}

Potřebu nových pojmenování lze naplnit různými způsoby, prvním z~nich je
využití již vytvořených pojmenování z~cizích jazyků -- k~tomuto
přejímání může docházet z~vícero důvodů, například když nemá domácí
jazyk pro daný pojem z~nějaké příčiny výraz (\emph{whiskey}), ale také
může jít o~nějakou formu jazykové módy (\emph{cool}) nebo eufemismu
(\emph{toaleta}).~\parencite[19]{dokulil62}

Druhým způsobem, jak vytvářet nová pojmenování, je tvorba nových výrazů
z~vlastní slovní zásoby daného jazyka. Podle Dokulila lze v~této oblasti
mluvit o~tvoření pojmenování víceslovných a~jednoslovných.
U~víceslovných novotvarů jde typicky jen o~novou kombinaci slov již
existujících (\emph{strojový překlad}), tento typ tvorby nových
pojmenování je v~českém jazyce nejčastější a~nejjednodušší, nicméně
existují důkazy, že čeština dává z~důvodu své flektivní povahy přednost
právě druhému typu. Ten je založen na tom, že nová jednoslovná
pojmenování vznikají prostřednictvím morfologických změn ze slov už
vytvořených -- může se tak jednat buď o~skládání slov (kompozici) nebo
odvozování slov (derivaci).~\parencite[21]{dokulil62}

\hypertarget{kompozice}{%
\subsection{Kompozice}\label{kompozice}}

Charakteristickým rysem výrazů vzniklých skládáním slov je to, že
obsahují více než jeden slovní základ (\emph{čern-o-vlasý}), někdy tedy
bývá kompozice označována za přechodný způsob mezi tvořením slov
víceslovný a~jednoslovných. Za specifický typ kompozit bývají označovány
spřežky (nevlastní složeniny) -- tato slova vznikla spřáhnutím slov,
která se často objevovala v~nějak slovním spojení
(\emph{boj-e-schopný}). Jejich vlastností je, že je lze za určitých
podmínek zpětně rozpojit do separátního stavu (\emph{schopný boje}).
\parencite[22]{dokulil62}

Ostatní složená slova jsou nerozložitelná do víceslovného pojmenování
a~jejich první člen nebývá hodnocen jako úplný tvar slova, taktéž bývá
mezi tvary přítomen kompoziční vokál \emph{o}, \emph{e} nebo \emph{i}.
Skládání slov není ve slovanských jazycích častým jevem, a~proto
i~v~češtině za nejdůležitější postup ve slovotvorbě považujeme derivaci.
\parencite[22]{dokulil62}

\hypertarget{derivace}{%
\subsection{Derivace}\label{derivace}}

Odvozování slov je založeno na tvoření slov od slov jiných (označujeme
je jako základové) prostřednictvím změny v~morfologické stavbě -- tyto
změny bývají způsobeny určitými odvozovacími prostředky (formanty).
\parencite[93]{dokulil00}

Podle pozice jednotlivých formantů můžeme vydělit několik základních
slovotvorných postupů:

\begin{itemize}
\tightlist
\item
  prefixace -- před slovo základové je umístěn slovotvorný morfém
  (prefix), jenž kromě slovesného vidu nemění mluvnické charakteristiky
  (\emph{ne-vinný}),
\item
  sufixace -- za slovo základové je umístěn slovotvorný morfém (sufix),
  ten se vždy připojuje za základ slova, nikoliv za celé základové slovo
  (\emph{prav-ic-e}). Sufixace je nejdůležitější slovotvorný postup
v~češtině~\parencite[23]{dokulil62} a~je spjata se souborem koncovek
  určitého paradigmatu~\parencite[93]{dokulil00},
\item
  reflexivizace -- odvozování zvratných sloves pomocí volných zvratných
  formantů \emph{se}, \emph{si} (\emph{bavit se}),
\item
  postfixace -- odvozování od úplného slovního tvaru pomocí zvláštní
  přípony (\emph{koho-si}),
\item
  deprefixace -- odsunutí prefixu (\emph{poslat} --\textgreater{}
  \emph{slát}),
\item
  desufiaxce -- odsunutí sufixu (\emph{plamen} --\textgreater{}
  \emph{plan}).~\parencite[93--94]{dokulil00}
\end{itemize}

Na závěr této podkapitoly je nutné poznamenat, že se výše vypsané
slovotvorné způsoby mohou různě kombinovat~\parencite[93]{dokulil00},
například podstatné jméno \emph{výsadek} je odvozeno od slova \emph{sad}
nebo \emph{sázet} jak pomocí prefixu \emph{vý}, tak prostřednictvím
sufixu -ek.

\hypertarget{lexikuxe1lnuxed-a-strukturuxe1lnuxed-vuxfdznam}{%
\section{Lexikální a~strukturální
význam}\label{lexikuxe1lnuxed-a-strukturuxe1lnuxed-vuxfdznam}}

\hypertarget{slovotvornuxe9-vztahy}{%
\section{Slovotvorné vztahy}\label{slovotvornuxe9-vztahy}}

\hypertarget{fundace}{%
\subsection{Fundace}\label{fundace}}

\hypertarget{motivace}{%
\subsection{Motivace}\label{motivace}}

\hypertarget{onomaziologickuxe9-kategorie-a-jejich-klasifikace}{%
\section{Onomaziologické kategorie a~jejich
klasifikace}\label{onomaziologickuxe9-kategorie-a-jejich-klasifikace}}

\hypertarget{substance-vlastnost-dux11bj-okolnost}{%
\subsection{Substance, vlastnost, děj,
okolnost}\label{substance-vlastnost-dux11bj-okolnost}}

\hypertarget{mutace}{%
\subsection{Mutace}\label{mutace}}

\hypertarget{modifikace}{%
\subsection{Modifikace}\label{modifikace}}

\hypertarget{transpozice}{%
\subsection{Transpozice}\label{transpozice}}

\hypertarget{slovotvornuxe9-tux159uxeddy-a-typy}{%
\section{Slovotvorné třídy
a~typy}\label{slovotvornuxe9-tux159uxeddy-a-typy}}

\hypertarget{definice}{%
\subsection{Definice}\label{definice}}

\hypertarget{dux11blenuxed}{%
\subsection{Dělení}\label{dux11blenuxed}}

\hypertarget{nuxe1stroje-pro-pruxe1ci-s-derivacuxed-v-ux10deskuxe9m-prostux159eduxed}{%
\chapter{Nástroje pro práci s~derivací v~českém
prostředí}\label{nuxe1stroje-pro-pruxe1ci-s-derivacuxed-v-ux10deskuxe9m-prostux159eduxed}}

První část této kapitoly popisuje nejvýznamnější softwarové nástroje,
které se využívají pro práci s~derivací v~českém prostředí. V~druhé
části si hlouběji představíme derivační síť Derinet, na níž je postaveno
řešení praktické části této bakalářské práce.

\hypertarget{pux159ehled-nuxe1strojux16f}{%
\section{Přehled nástrojů}\label{pux159ehled-nuxe1strojux16f}}

\hypertarget{derivancze}{%
\subsection{Derivancze}\label{derivancze}}

Něco hezkého~\parencite[516]{pala15}

\hypertarget{morfio}{%
\subsection{Morfio}\label{morfio}}

\hypertarget{ajka}{%
\subsection{Ajka}\label{ajka}}

\hypertarget{deriv}{%
\subsection{Deriv}\label{deriv}}

\hypertarget{derivaux10dnuxed-suxedux165-derinet}{%
\section{Derivační síť
Derinet}\label{derivaux10dnuxed-suxedux165-derinet}}

\part{Praktická část}

\hypertarget{zpracovanuxe9-slovotvornuxe9-sufixy}{%
\chapter{Zpracované slovotvorné
sufixy}\label{zpracovanuxe9-slovotvornuxe9-sufixy}}

Cílem bakalářské práce bylo vytvořit elektronický slovník s~definicemi
založenými na derivačních rysech slovotvorně motivovaných slov ve formě
mobilní aplikace, proto si v~této kapitole praktické části popíšeme,
jaké slovotvorné sufixy byly zpracovány a~jakým způsobem probíhá proces
tvoření slovotvorných definic. V~druhé kapitole si pak rozebereme
technickou část spolu s~použitými technologiemi a~popisem implementace
samotné mobilní aplikace.

Při výběru slovotvorných typů ke zpracování byl brán zřetel na jejich
frekvenci a~produktivitu ve spojitosti s~derivací u~substantiv --
takovým nejvýraznějším typem jsou právě substantiva označující názvy
živých bytostí podle jejich činností zakončená na sufix \emph{-tel},
a~ta spadají do slovotvorné třídy činitelských jmen.
\parencite[17]{dokulil67}

\hypertarget{slovotvornuxfd-typ--tel}{%
\section{Slovotvorný typ -tel}\label{slovotvornuxfd-typ--tel}}

Pro naše účely bylo důležité vybrat takový slovotvorný typ, u~něhož se
převážně shoduje slovotvorný a~lexikální význam. Tato podmínka je
splněna, protože z~výzkumné práce Adriany Válkové (To jsem šplhoun, co?
XXX) vyplývá, že z~1 129 zkoumaných lemmat zakončených příponou
\emph{-tel} pouze u~3,03~\% z~nich lexikální význam nahrazuje význam
strukturní (to se týká převážně neživotných substantiv typu
\emph{jmenovatel}, \emph{dělitel} atd.) a~celkově je u~4,42~\% lexikální
význam obecnější
(např.\emph{vnímatel}\footnote{význam slova *vnímatel* je dle SSJČ „kdo uvědoměle vnímá umělecké dílo“~\parencite{ssjc}, zde došlo prokazatelně k~lexikalizaci slovotvorného významu.}).
Tento lingvistický výzkum byl založen na datech z~korpusu SYNv6
a~jednotlivé typy významů byly porovnávány prostřednictvím výkladových
slovníků. citace\{Adri-XXX\}

Jak bylo již naznačeno, obecný význam činitelských jmen je podle
Dokulila „názvy osob a~živých bytostí vůbec podle povahy a~druhu jejich
činností``~\parencite[17]{dokulil67}. Sufix \emph{-tel} tak vyjadřuje,
že takto odvozený pojem je subjektem děje základového slovesa, s~tím že
nejčastěji jde o~aktivní\footnote{To neplatí u~substantiv *trpitel*, *truchlitel* a~*bydlitel*, které jsou odvozený ze stavových sloves.~\parencite[17]{dokulil67}}
účast subjektu na ději (např. subjekt označen slovem \emph{učitel}
vykonává takovou činnost, kterou vyjadřuje sloveso \emph{učit}, z~něhož
je výraz odvozený). U~tohoto slovotvorného typu se typicky jedná
o~mužská substantiva, nicméně se najdou i~výjimky v~podobě neživotných
substantiv (viz předchozí odstavec).~\parencite{simandl2016}

Nejčastěji jsou substantiva tohoto slovotvorného typu odvozena od
imperfektiv, ze zkoumaných 1129 lemmat je jich takto derivováno
přibližně 74,3~\%, navíc se určitá část substantiv vzniklých z~perfektiv
chová, jako byly odvozeny z~imperfektiv, jde typicky o~názvy profesí
(\emph{zastoupit} --\textgreater{} \emph{zastupitel}) a~názvy osob, pro
které je daná činnost typická, ale nejsou označovány za samostatné
profese. (\emph{chovat} --\textgreater{} \emph{chovatel}).
citace\{Adri-XXX\}

\hypertarget{elektronickuxfd-derivaux10dnuxed-slovnuxedk}{%
\chapter{Elektronický derivační
slovník}\label{elektronickuxfd-derivaux10dnuxed-slovnuxedk}}

V~následující kapitole si představíme a~následně popíšeme výsledek
praktické části, a~to nejprve v~krátkosti po motivační stránce a~posléze
po stránce technické.

Derivační slovník je primárně koncipován jako edukační pomůcka pro
cizince, kteří se učí češtinu jako druhý jazyk. Na rozdíl od rodilých
mluvčí nedokáží cizinci podvědomě predikovat význam neznámých slov na
základě slovotvorných morfému v~určitých kontextech -- chybí jim tedy
znalost významů určitých slovotvorných afixů, prostřednictvím kterých by
si pak dokázali analogicky vyvodit význam slova neznámého.

Díky informacím z~tohoto slovníku by tak studující mohli být schopni
odhadnout významy například takových internacionalismů, které byly
přejaty do slovotvorného systému českého jazyka pomocí sufixů. Taktéž se
očekává intuitivnější chápání derivačních pravidel u~cizinců, jejichž
rodný jazyk patří do skupiny slovanských jazyků (z~důvodu flektivního
charakteru těchto jazyků).

\hypertarget{poux17eadavky-na-aplikaci}{%
\section{Požadavky na aplikaci}\label{poux17eadavky-na-aplikaci}}

Primárním zadáním praktické části bylo vytvořit derivační slovník ve
formě mobilní aplikace, který bude využívat slovotvorných informací
z~derivační sítě DeriNet. Dalším požadavkem, který vychází přímo z~povahy
samotného slovníku jakožto podpůrného nástroje pro výuku cizinců, bylo
vyhledat a~implementovat dvojjazyčný česko-anglický slovník, a~to proto,
aby byla celá aplikace včetně slovotvorných definic kompletně
lokalizovaná v~anglickém jazyce.

Požadavky na funkcionalitu slovníku jako takového můžeme ve stručnosti
shrnout v~několika bodech:

\begin{itemize}
\tightlist
\item
  funkce \emph{insert word} --\textgreater{} vrátí se zadaného vstupu:

  \begin{itemize}
  \tightlist
  \item
    částečnou slovotvornou analýzu;
  \item
    anglickou i~českou definici založenou na strukturním významu slova
    (v~případě že se takový ekvivalent bude nacházet ve vybraném
    česko-anglickém slovníku);
  \item
    doplňující derivační a~morfologické informace;
  \end{itemize}
\item
  heslář již zpracovaných slov ve formě rejstříku.
\end{itemize}

Součástí zadání také bylo to, aby všechny funkcionality mobilní aplikace
byly kompletně funkční bez připojení k~internetu -- tím párem nebylo
zapotřebí řešit autentifikaci
uživatele\footnote{Nicméně je tato možnost stále v~řešení, a~to pro případ, kdybychom v~aplikaci chtěli nabídnout možnost ukládání již naučených hesel do osobního adresáře atd.}
či pracovat se vzdálenými uložištěmi.

\hypertarget{nuxe1vrh-aplikace}{%
\section{Návrh aplikace}\label{nuxe1vrh-aplikace}}

Na začátku samotného vývoje si je zapotřebí určit několik věci, v~našem
případě jde primárně o:

\begin{itemize}
\tightlist
\item
  zvolení vhodných technologií včetně programovacího jazyka, kterými
  budeme nástroj implementovat;
\item
  výběr dat a~jejich struktury, nad kterými budeme v~rámci aplikace
  operovat;
\item
  návrh jednotlivých obrazovek aplikace a~navigaci mezi nimi (včetně
  konkrétních přechodů).
\end{itemize}

\hypertarget{pouux17eituxe9-technologie}{%
\subsection{Použité technologie}\label{pouux17eituxe9-technologie}}

Tradiční způsob vývoje mobilních aplikací se obecně dělí na tři hlavní
typy -- jde o~takzvané webové, nativní a~hybridní aplikace. Každý
z~těchto přístupů má svá vlastní pozitiva a~negativa, a~tedy si je
zapotřebí na začátku každého vývoje určit, pro jaké účely má daná
aplikace sloužit a~jaké funkcionality splňovat.

Webové aplikace fungují typicky na všech platformách a~jsou založeny na
klasických webových technologiích, tzn. na HTML, CSS a~na programovacím
jazyce JavaScript (viz další kapitola), jedná se tedy o~přizpůsobené
webové stránky, z~čehož vyplývá potřeba internetového připojení. Výhodou
tohoto přístupu je kromě již zmíněné multiplatformní povahy ukládání dat
na webových serverech, tyto aplikace tak nevyžadují velké množství
paměti na lokálním uložišti. Za hlavní negativum je považována nižší
kompatibilita s~hardwarem a~operačním systémem u~daných mobilních
zařízení.

Na druhou stranu nativní aplikace jsou vytvořeny pouze pro jednu
specifickou platformu, to znamená, že například aplikaci vytvořenou pro
systém Android nelze spustit na systému iOS a~naopak. Z~tohoto přístupu
vyplývají výhody ve formě maximálního využití daného operačního systému
(větší výkon, kompatibilita, uživatelská zkušenosti, \ldots{}), ale
i~nevýhody týkající se nutnosti využívání specifických technologií
určitého operačního systému (například pro operační systém iOS se
využívá programovací jazyk Objective-C (nově Swift), pro Android je
určen jazyk Java).

Posledním typem jsou pak hybridní aplikace, které kombinují oba předešlé
přístupy -- vývoj probíhá ve specializovaném nástroji za použití
webových technologií, v~rámci kterého se testuje logika jednotlivých
funkcionalit. Po jeho dokončení dochází ke kompilaci do vybraného
operačního systému, v~rámci kterého se již pracuje s~klasickými
nativními funkcemi (například s~mikrofonem, fotoaparátem, lokálním
uložištěm v~telefonu atd.). Výhoda hybridních aplikací je
v~jednoduchosti vývoje, který je v~porovnání s~nativním rychlý a~snadný,
a~to i~z~toho důvodu, že se u~něj pracuje pouze s~jedním zdrojem kódu,
který je použitelný na větším počtu platforem. Negativním aspektem je
pak pomalejší výpočetní výkon, který je spojen s~využitím speciálních
knihoven pro převod do výsledné mobilní aplikace.

Jelikož naše mobilní aplikace používá minimum nativních funkcí a~jde nám
spíše o~rozšíření nástroje napříč různými platformami, zvolíme hybridní
formu vývoje.

\hypertarget{webovuxe9-technologie}{%
\subsubsection{Webové technologie}\label{webovuxe9-technologie}}

\hypertarget{frameworky-ionic-a-angular}{%
\subsubsection{Frameworky Ionic
a~Angular}\label{frameworky-ionic-a-angular}}

\hypertarget{volba-dat}{%
\subsection{Volba dat}\label{volba-dat}}

\hypertarget{nuxe1vrh-obrazovek}{%
\subsection{Návrh obrazovek}\label{nuxe1vrh-obrazovek}}

\hypertarget{implementace-aplikace}{%
\section{Implementace aplikace}\label{implementace-aplikace}}

\include{tex/zaver}
\clearpage

\pagestyle{plain}

\addcontentsline{toc}{chapter}{Seznam literatury}
\begin{spacing}{1.05}
\printbibliography[title={Seznam literatury}]
\end{spacing}

\end{document}