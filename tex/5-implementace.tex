\hypertarget{elektronickuxfd-derivaux10dnuxed-slovnuxedk}{%
\chapter{Elektronický derivační
slovník}\label{elektronickuxfd-derivaux10dnuxed-slovnuxedk}}

V~následující kapitole si představíme a~následně popíšeme výsledek
praktické části, a~to nejprve v~krátkosti po motivační stránce a~posléze
po stránce technické.

Derivační slovník je primárně koncipován jako edukační pomůcka pro
cizince, kteří se učí češtinu jako druhý jazyk. Na rozdíl od rodilých
mluvčí nedokáží cizinci podvědomě predikovat význam neznámých slov na
základě slovotvorných morfému v~určitých kontextech -- chybí jim tedy
znalost významů určitých slovotvorných afixů, prostřednictvím kterých by
si pak dokázali analogicky vyvodit význam slova neznámého.

Díky informacím z~tohoto slovníku by tak studující mohli být schopni
odhadnout významy například takových internacionalismů, které byly
přejaty do slovotvorného systému českého jazyka pomocí sufixů. Taktéž se
očekává intuitivnější chápání derivačních pravidel u~cizinců, jejichž
rodný jazyk patří do skupiny slovanských jazyků (z~důvodu flektivního
charakteru těchto jazyků).

\hypertarget{poux17eadavky-na-aplikaci}{%
\section{Požadavky na aplikaci}\label{poux17eadavky-na-aplikaci}}

Primárním zadáním praktické části bylo vytvořit derivační slovník ve
formě mobilní aplikace, který bude využívat slovotvorných informací
z~derivační sítě DeriNet. Dalším požadavkem, který vychází přímo z~povahy
samotného slovníku jakožto podpůrného nástroje pro výuku cizinců, bylo
vyhledat a~implementovat dvojjazyčný česko-anglický slovník, a~to proto,
aby byla celá aplikace včetně slovotvorných definic kompletně
lokalizovaná v~anglickém jazyce.

Požadavky na funkcionalitu slovníku jako takového můžeme ve stručnosti
shrnout v~několika bodech:

\begin{itemize}
\tightlist
\item
  funkce \emph{insert word} --\textgreater{} vrátí se zadaného vstupu:

  \begin{itemize}
  \tightlist
  \item
    částečnou slovotvornou analýzu;
  \item
    anglickou i~českou definici založenou na strukturním významu slova
    (v~případě že se takový ekvivalent bude nacházet ve vybraném
    česko-anglickém slovníku);
  \item
    doplňující derivační a~morfologické informace;
  \end{itemize}
\item
  heslář již zpracovaných slov ve formě rejstříku.
\end{itemize}

Součástí zadání také bylo to, aby všechny funkcionality mobilní aplikace
byly kompletně funkční bez připojení k~internetu -- tím párem nebylo
zapotřebí řešit autentifikaci
uživatele\footnote{Nicméně je tato možnost stále v~řešení, a~to pro případ, kdybychom v~aplikaci chtěli nabídnout možnost ukládání již naučených hesel do osobního adresáře atd.}
či pracovat se vzdálenými uložištěmi.

\hypertarget{nuxe1vrh-aplikace}{%
\section{Návrh aplikace}\label{nuxe1vrh-aplikace}}

Na začátku samotného vývoje si je zapotřebí určit několik věci, v~našem
případě jde primárně o:

\begin{itemize}
\tightlist
\item
  zvolení vhodných technologií včetně programovacího jazyka, kterými
  budeme nástroj implementovat;
\item
  výběr dat a~jejich struktury, nad kterými budeme v~rámci aplikace
  operovat;
\item
  návrh jednotlivých obrazovek aplikace a~navigaci mezi nimi (včetně
  konkrétních přechodů).
\end{itemize}

\hypertarget{pouux17eituxe9-technologie}{%
\subsection{Použité technologie}\label{pouux17eituxe9-technologie}}

Tradiční způsob vývoje mobilních aplikací se obecně dělí na tři hlavní
typy -- jde o~takzvané webové, nativní a~hybridní aplikace. Každý
z~těchto přístupů má svá vlastní pozitiva a~negativa, a~tedy si je
zapotřebí na začátku každého vývoje určit, pro jaké účely má daná
aplikace sloužit a~jaké funkcionality splňovat.

Webové aplikace fungují typicky na všech platformách a~jsou založeny na
klasických webových technologiích, tzn. na HTML, CSS a~na programovacím
jazyce JavaScript (viz další kapitola), jedná se tedy o~přizpůsobené
webové stránky, z~čehož vyplývá potřeba internetového připojení. Výhodou
tohoto přístupu je kromě již zmíněné multiplatformní povahy ukládání dat
na webových serverech, tyto aplikace tak nevyžadují velké množství
paměti na lokálním uložišti. Za hlavní negativum je považována nižší
kompatibilita s~hardwarem a~operačním systémem u~daných mobilních
zařízení.

Na druhou stranu nativní aplikace jsou vytvořeny pouze pro jednu
specifickou platformu, to znamená, že například aplikaci vytvořenou pro
systém Android nelze spustit na systému iOS a~naopak. Z~tohoto přístupu
vyplývají výhody ve formě maximálního využití daného operačního systému
(větší výkon, kompatibilita, uživatelská zkušenosti, \ldots{}), ale
i~nevýhody týkající se nutnosti využívání specifických technologií
určitého operačního systému (například pro operační systém iOS se
využívá programovací jazyk Objective-C (nově Swift), pro Android je
určen jazyk Java).

Posledním typem jsou pak hybridní aplikace, které kombinují oba předešlé
přístupy -- vývoj probíhá ve specializovaném nástroji za použití
webových technologií, v~rámci kterého se testuje logika jednotlivých
funkcionalit. Po jeho dokončení dochází ke kompilaci do vybraného
operačního systému, v~rámci kterého se již pracuje s~klasickými
nativními funkcemi (například s~mikrofonem, fotoaparátem, lokálním
uložištěm v~telefonu atd.). Výhoda hybridních aplikací je
v~jednoduchosti vývoje, který je v~porovnání s~nativním rychlý a~snadný,
a~to i~z~toho důvodu, že se u~něj pracuje pouze s~jedním zdrojem kódu,
který je použitelný na větším počtu platforem. Negativním aspektem je
pak pomalejší výpočetní výkon, který je spojen s~využitím speciálních
knihoven pro převod do výsledné mobilní aplikace.

Jelikož naše mobilní aplikace používá minimum nativních funkcí a~jde nám
spíše o~rozšíření nástroje napříč různými platformami, zvolíme hybridní
formu vývoje.

\hypertarget{webovuxe9-technologie}{%
\subsubsection{Webové technologie}\label{webovuxe9-technologie}}

\hypertarget{frameworky-ionic-a-angular}{%
\subsubsection{Frameworky Ionic
a~Angular}\label{frameworky-ionic-a-angular}}

\hypertarget{volba-dat}{%
\subsection{Volba dat}\label{volba-dat}}

\hypertarget{nuxe1vrh-obrazovek}{%
\subsection{Návrh obrazovek}\label{nuxe1vrh-obrazovek}}

\hypertarget{implementace-aplikace}{%
\section{Implementace aplikace}\label{implementace-aplikace}}
