\hypertarget{elektronickuxfd-derivaux10dnuxed-slovnuxedk}{%
\chapter{Elektronický derivační
slovník}\label{elektronickuxfd-derivaux10dnuxed-slovnuxedk}}

V~této kapitole popisujeme výsledek praktické části, a~to jak z~hlediska
formálních požadavků na nástroj jako takový, tak z~hlediska procesu
návrhu a~implementace.

Derivační slovník je primárně koncipován jako edukační pomůcka pro
cizince, kteří se učí češtinu jako druhý jazyk. Na rozdíl od rodilých
mluvčí nedokážou cizinci podvědomě predikovat význam neznámých slov na
základě slovotvorných morfému v~určitých kontextech. Chybí jim tedy
podvědomá znalost významů určitých slovotvorných afixů, prostřednictvím
kterých by si pak dokázali analogicky vyvodit význam slova neznámého.

Věříme, že by se tento slovník mohl stát dobrou učební pomůckou, pomocí
které by mohli cizinci porozumět např. takovým internacionalismům, které
byly přejaty do slovotvorného systému českého jazyka pomocí sufixů.

Díky pravidelnému využívání slovníku taktéž očekáváme zejména u~cizinců
se slovanským rodným jazykem (z~důvodu flektivního charakteru těchto
jazyků) rychlejší akvizici derivačních vztahů a~principů v~češtině.

\hypertarget{poux17eadavky-na-aplikaci}{%
\section{Požadavky na aplikaci}\label{poux17eadavky-na-aplikaci}}

Cílem práce bylo vytvořit derivační slovník ve formě mobilní aplikace,
který bude využívat slovotvorných informací z~derivační sítě DeriNet.
Dílčím požadavkem, který vychází přímo z~povahy samotného slovníku
jakožto podpůrného nástroje pro výuku cizinců, bylo vyhledat
a~implementovat dvojjazyčný česko-anglický slovník, a~to proto, aby byla
celá aplikace včetně slovotvorných definic kompletně lokalizovaná
v~anglickém jazyce.

Požadavky na funkcionalitu slovníku jako takového můžeme ve stručnosti
shrnout v~následujících bodech:

\begin{itemize}
\tightlist
\item
  funkce \emph{insert word} $\rightarrow$ vrátí se zadaného vstupu:

  \begin{itemize}
  \tightlist
  \item
    částečnou slovotvornou analýzu;
  \item
    anglickou i~českou definici založenou na strukturním významu slova
    (anglickou pouze v~případě, že se takový ekvivalent bude nacházet ve
    vybraném česko-anglickém slovníku);
  \item
    doplňující derivační a~morfologické informace;
  \end{itemize}
\item
  heslář již zpracovaných slov ve formě rejstříku.
\end{itemize}

Součástí zadání také bylo to, aby všechny funkcionality mobilní aplikace
byly kompletně funkční bez připojení k~internetu -- tím pádem nebylo
zapotřebí řešit autentifikaci
uživatele\footnote{Nicméně je tato možnost stále v~řešení, a~to pro případ, kdybychom v~aplikaci chtěli nabídnout možnost ukládání již naučených hesel do osobního adresáře atd.}
či pracovat se vzdáleným uložištěm.

\hypertarget{nuxe1vrh-aplikace}{%
\section{Návrh aplikace}\label{nuxe1vrh-aplikace}}

Na začátku samotného vývoje si je zapotřebí určit několik věci, v~našem
případě jde primárně o:

\begin{itemize}
\tightlist
\item
  zvolení vhodných technologií včetně programovacího jazyka, kterými
  budeme nástroj implementovat;
\item
  návrh uživatelského rozhraní aplikace spolu s~navigací mezi
  jednotlivými vizuálními prvky (včetně konkrétních přechodů).
\end{itemize}

\hypertarget{pouux17eituxe9-technologie}{%
\subsection{Použité technologie}\label{pouux17eituxe9-technologie}}

Tradiční způsob vývoje mobilních aplikací se obecně dělí na tři hlavní
typy -- jde o~takzvané webové, nativní a~hybridní aplikace. Každý
z~těchto přístupů má svá vlastní pozitiva a~negativa, a~tedy si je
zapotřebí na začátku každého vývoje určit, pro jaké účely má daná
aplikace sloužit a~jaké funkcionality splňovat.

Webové aplikace fungují typicky na všech platformách a~jsou založeny na
klasických webových technologiích, tzn. na HTML, CSS a~na programovacím
jazyce JavaScript (viz podkapitola \ref{webovuxe9-technologie}), jedná
se tedy o~přizpůsobené webové stránky, z~čehož vyplývá potřeba
internetového připojení. Výhodou tohoto přístupu je kromě již zmíněné
multiplatformní povahy ukládání dat na webových serverech, tyto aplikace
tak nevyžadují velké množství paměti na lokálním uložišti. Za hlavní
negativum je považována nižší kompatibilita s~hardwarem a~operačním
systémem u~daných mobilních zařízení.

Na druhou stranu nativní aplikace jsou vyvíjeny pouze pro jednu
specifickou platformu, to znamená, že například aplikaci vytvořenou pro
systém Android nelze spustit na systému iOS a~naopak. Z~tohoto přístupu
vyplývají výhody ve formě maximálního využití daného operačního systému
(větší výkon, kompatibilita, uživatelská zkušenosti, \ldots{}), ale
i~nevýhody týkající se nutnosti využívání specifických technologií
určitého operačního systému (například pro operační systém iOS se
využívá programovací jazyk Objective-C (nově Swift), pro Android je
určen jazyk Java).

Posledním typem jsou pak hybridní aplikace, které kombinují oba předešlé
přístupy -- vývoj probíhá ve specializovaném nástroji za použití
webových technologií, v~rámci kterého se testuje logika jednotlivých
funkcionalit. Po jeho dokončení dochází ke kompilaci do vybraného
operačního systému, v~rámci kterého se již pracuje s~klasickými
nativními funkcemi (například s~mikrofonem, fotoaparátem, lokálním
uložištěm v~telefonu atd.). Výhoda hybridních aplikací je
v~jednoduchosti vývoje, který je v~porovnání s~nativním rychlý a~snadný,
a~to i~z~toho důvodu, že se u~něj pracuje pouze s~jedním zdrojem kódu,
který je však použitelný na větší počet platforem. Negativním aspektem
je pak pomalejší výpočetní výkon, který je spojen s~využitím speciálních
knihoven pro převod do výsledné podoby mobilní aplikace.

Jelikož se v~našem případě pracuje s~minimem nativních funkcí, aplikace
by měla být nezávislá na internetovém připojení a~jde nám spíše
o~rozšíření nástroje napříč různými platformami, zvolíme hybridní formu
vývoje.

\hypertarget{webovuxe9-technologie}{%
\subsubsection{Webové technologie}\label{webovuxe9-technologie}}

Jak bylo výše poznamenáno, vývoj hybridních aplikací probíhá
prostřednictvím klasických webových technologií, do nichž spadá HTML,
CSS a~JavaScript, proto si je v~následujících odstavcích v~krátkosti
popíšeme.

HTML (Hypertext Markup Language) a~CSS (Cascading Style Sheets) jsou dvě
základní technologie, na nichž jsou v~nejobecnější rovině postaveny
webové stránky. HTML je značkovací jazyk, pomocí kterého popisujeme
strukturu webových stránek, což znamená, že za použití speciálně
definovaných značek přiřazujeme jednotlivé významy částem obsahu. Pomocí
CSS pak nastavujeme vizuální podobu dané webové stránky (barvy, fonty,
odsazení atd.). Důležitou vlastností CSS je nezávislost na HTML, můžeme
tedy dle potřeby tyto dva jazyky separovat a~následně s~mírnými obměnami
využít v~rámci jiného projektu/prostředí.~\parencite{htmlcss}

Před samotným vykreslením určité webové stránky musí dojít k~převedení
HTML dokumentu do objektu zvaného DOM (Document Object Model) -- jde
o~HTML uložené ve stromové struktuře, kde je o~každém jednotlivém uzlu
(tedy značce) uložená informace o~jeho lokaci. S~tímto objektem dále
pracuje CSS (aplikují se jednotlivá pravidla týkající se vykreslení)
a~JavaScript, jehož prostřednictvím se pak může dynamicky měnit struktura
a~zobrazení tohoto stromového objektu.~\parencite{howbrowserswork}

JavaScript (obecněji ECMAscript) je
skriptovací\footnote{Jako skriptovací jazyk je označován z~toho důvodu, že není při svém spuštění nijak kompilován (například na rozdíl od programovacích jazyků jako jsou Java nebo Objective-C) a~namísto toho je rovnou proveden daným webovým prohlížečem (ten je tak jeho interpretem).}
programovací jazyk, který je součástí všech moderních webových
prohlížečů. JavaScript se používá na takových místech, kde je zapotřebí
v~rámci webové stránky určitým způsobem interagovat s~jejím uživatelem,
to znamená, že ho typicky využijeme v~oblasti webových aplikací, u~nichž
je očekávaný zásah uživatele a~je zapotřebí dynamicky měnit zobrazovaný
obsah.~\parencite{javascript}

\hypertarget{framework-ionic-a-angular}{%
\subsubsection{Framework Ionic a~Angular}\label{framework-ionic-a-angular}}

Při vývoji komplexnějších webových a~mobilních aplikací se často pracuje
s~takzvanými webovými frameworky, které určitým způsobem zjednodušují
vývoj. Ve stručnosti je jejich účelem umožnit vývojářům práci v~takovém
prostředí, v~němž se mohou primárně zaměřit na požadavky daného projektu
a~nemusí řešit problematiku opakujících se funkcí nebo závislosti
jednotlivých knihoven.

My jsme v~rámci naší mobilní aplikace využili Ionic Framework, což je
soubor nástrojů s~otevřeným zdrojovým kódem pod licencí
MIT\footnote{https://opensource.org/licenses/MIT}, který se zaměřuje na
uživatelské rozhraní u~mobilních a~webových aplikací. Obsahuje velké
množství vizuálních komponent, jež jsou typické pro mobilní zařízení,
a~taktéž je zde umožněno pomocí knihovny Ionic Native přistupovat
k~nativním funkcím mobilního zařízení -- zde se využívá knihovna Cordova,
která slouží pro převod webové aplikace do aplikace mobilní.
\parencite{cordova}

Dalším cílem Ionicu je poskytnout takové prostředí pro vývoj hybridních
aplikací, které je kompatibilní s~dalšími frameworky, jako jsou
například React nebo Vue, nicméně jeho jádro a~struktura samotného kódu
je založena na webovém frameworku Angular.~\parencite{ionic}

Angular jako takový pracuje s~programovacím jazykem TypeScript, který je
vyvíjen firmou Microsoft jako open-source. TypeScript je rozšíření již
zmíněného JavaScriptu, to v~praxi znamená, že kód napsaný v~TypeScriptu
je zapotřebí nejdříve převést do určité podoby javaScriptového standardu
a~ten je až poté možné interpretovat ve webových prohlížečích. Tento
programovací jazyk se používá primárně z~důvodu své silné typové
kontroly. Pokud je tedy této možnosti využito, lze tak díky ní
předcházet mnohým potenciálním chybám.~\parencite{typescript}

Architektura frameworku Angular se skládá z~mnoha vzájemně provázaných
vrstev, ve stručnosti popíšeme právě ty, které jsou pro pochopení naši
aplikace klíčové. Základním prvkem architektury jsou \emph{komponenty},
v~rámci kterých dochází k~propojení aplikační logiky s~konkrétní HTML
šablonou a~CSS (ty společně určují celkový vzhled). Jedna komponenta je
většinou rovna jedné vizuální stránce v~aplikaci, nicméně Angular dokáže
svými vlastními značkami (\emph{direktivami}) ovlivnit výslednou podobou
šablony ještě před jejím zobrazením -- díky tomuto principu jednoduše
dochází k~dynamické proměně obsahu stránek podle zadané aplikační
logiky, protože ta ovlivňuje chování samotných direktiv. Další důležitou
části jsou \emph{služby} -- ty už nejsou propojeny s~žádnou vizuální
stránkou (s~HTML šablonou), ale slouží pro dekompozici jednotlivých
funkcionalit napříč aplikací. Data, která jsou těmito službami
zpracována, jsou pak dále přeposílána do určitých komponent.
\parencite{angulararchitecture}

\hypertarget{nuxe1vrh-uux17eivatelskuxe9ho-rozhranuxed}{%
\subsection{Návrh uživatelského
rozhraní}\label{nuxe1vrh-uux17eivatelskuxe9ho-rozhranuxed}}

V~této podkapitole popíšeme hlavní vizuální prvky uživatelského rozhraní
a~zaměříme se zde na popis chování rozhraní při interakci s~uživatelem.
Návrh celého uživatelského rozhraní přímo vychází z~požadavků na
aplikaci jako takovou (viz \ref{poux17eadavky-na-aplikaci}) a~je
demonstrován na mobilním telefonu s~operačním systémem Android.

První obrazovka, se kterou uživatel přichází do styku po spuštění
aplikace, je úvodní stránka, která uvádí základní informace o~derivačním
slovníku (viz obrázek \ref{1}). Pod těmito informacemi se vyskytují dvě
velká navigační tlačítka (\emph{open index} a~\emph{search}), která
uživatele vedou k~využití hlavních funkcionalit aplikace.

\begin{figure}[ht]
  \begin{subfigure}[b]{0.45\textwidth}
    \includegraphics[width=0.9\textwidth]{1-1}
  \end{subfigure}
  \hfill
  \begin{subfigure}[b]{0.45\textwidth}
    \includegraphics[width=0.9\textwidth]{1-2}
  \end{subfigure}
  \caption{Úvodní stránka a~navigační menu}
  \label{1}
\end{figure}

Druhá možnost, jak využít navigačního systému aplikace, je pomocí
navigačního vysouvacího menu (viz obrázek \ref{1}), jež nabízí čtyři
možná přesměrování. Barevnou indikací je vždy označena aktuální pozice
uživatele v~systému -- tohle menu je vždy přístupné v~levém horním rohu
obrazovky.

Nejdůležitějším prvkem celého uživatelského rozhraní je přístup k~hlavní
funkcionalitě \emph{insert word}. Domníváme se, že právě zde musí dojít
k~co nejpříjemnějšímu uživatelskému zážitku, protože na tomto místě
dochází k~uspokojení, či neuspokojení potřeb uživatele vzhledem
k~aplikaci.

S~ohledem na výše zmíněné je výchozím stavem prázdné textové pole, které
očekává vstup od uživatele. Po zadání jednotlivých znaků obrazovka
dynamicky zobrazuje pouze ta slova z~rejstříku zpracovaných slov, která
začínají podřetězcem zadaného slovního tvaru (viz obrázek \ref{2}).
A~právě přes tato vyselektovaná slova se lze dostat k~výslednému
slovníkovému heslu.

\begin{figure}[ht]
  \begin{subfigure}[b]{0.3\textwidth}
    \includegraphics[width=\textwidth]{2-1}
  \end{subfigure}
  \hfill
  \begin{subfigure}[b]{0.3\textwidth}
    \includegraphics[width=\textwidth]{2-2}
  \end{subfigure}
   \hfill
  \begin{subfigure}[b]{0.3\textwidth}
   \includegraphics[width=\textwidth]{2-3}
  \end{subfigure}
  \caption{Proces zadávání vstupního slova}
  \label{2}
\end{figure}

Samotné slovníkové heslo je zobrazeno formou tří na sobě nezávislých
karet (viz obrázek \ref{3}), z~nichž každá obsahuje určité lingvistické
informace o~derivovaném výrazu (viz \ref{slovnuxedkovuxe9-heslo}).
V~případě, kdy je heslo vytvořeno na základě slova vybraného z~rejstříku
zpracovaných slov, může uživatel využít zobrazené navigační šipky (která
se zobrazí vedle vstupního výrazu po kliknutí na textové pole), jež ho
vrátí zpět na určité místo v~rejstříku.

\begin{figure}[ht]
  \begin{subfigure}[b]{0.45\textwidth}
    \includegraphics[width=0.9\textwidth]{3-1}
  \end{subfigure}
  \hfill
  \begin{subfigure}[b]{0.45\textwidth}
    \includegraphics[width=0.9\textwidth]{3-2}
  \end{subfigure}
  \caption{Slovníková hesla}
  \label{3}
\end{figure}

Rejstřík zpracovaných slov je řazen abecedně a~obsahuje úplný seznam
slov (viz obrázek \ref{4}), která spadají do zpracovaných slovotvorných
typů. Jak bylo výše naznačeno, prostřednictvím rejstříku může uživatel
taktéž přistoupit k~vygenerování určitého slovníkového hesla.

Zbývají obrazovka je už jen marginální informační stránka týkající se
autorů, jež není pro popis návrhu uživatelského rozhraní nijak důležitá.

\begin{figure}[ht]
  \begin{subfigure}[b]{0.45\textwidth}
    \includegraphics[width=0.9\textwidth]{4-1}
  \end{subfigure}
  \hfill
  \begin{subfigure}[b]{0.45\textwidth}
    \includegraphics[width=0.9\textwidth]{4-2}
  \end{subfigure}
  \caption{Rejstřík zpracovaných slov}
  \label{4}
\end{figure}

\hypertarget{implementace-aplikace}{%
\section{Implementace aplikace}\label{implementace-aplikace}}

V~poslední kapitole si popíšeme architekturu předkládané aplikace spolu
s~její hlavní funkcionalitou \emph{insert word.}

\hypertarget{architektura}{%
\subsection*{Architektura aplikace}\label{architektura}}

Mobilní aplikace se skládá z~pěti hlavních stránek (komponent), jde o:

\begin{itemize}
\tightlist
\item
  vysouvací menu;
\item
  úvodní obrazovku s~informacemi o~aplikaci;
\item
  stránku s~funkcionalitou \emph{insert word};
\item
  rejstřík se zpracovanými slovy;
\item
  stránku s~informacemi o~autorech.
\end{itemize}

\hypertarget{insert word}{%
\subsection*{Alogritmus funkcionality insert word}\label{insert word}}

Implementačně nejkomplexnější je komponenta s~funkcionalitou
\emph{insert word}, proto se na ní v~následující části důkladněji
zaměříme a~pro demonstraci použitého algoritmu použijeme přiložené
schéma (viz obrázek \ref{algoritmus}). Pro větší přehlednost jsou na
diagramu modře zvýrazněny komponenty, žlutou barvou služby, zeleně
interní uložiště s~daty a~červeně pak hlavní funkce (ty se dále větví do
menších podfunkcí, jejichž popis není pro účely tohoto popisu klíčový).

\begin{figure}[ht]   
    \centering
    \includegraphics[width=.9\textwidth]{algoritmus}  
    \caption{Algoritmus funkcionality insert word}
    \label{algoritmus}
 \end{figure}

V~prvé řadě musí aplikace nějakým způsobem získat vstup pro následnou
analýzu, existují dva způsoby, jak k~tomu docílit. V~případě kroku 1.a
je za vstup považováno takové slovo, které bylo vybráno v~rámci
komponenty \emph{index}, tedy je vstupní slovo vybráno na stránce
s~rejstříkem zpracovaných slov. Druhou možností (1.b) je zadat slovo ručně
prostřednictvím funkce \emph{fromUser} z~textového pole, v~takovém
případě se po zadání prvního písmena (a~následně dalších znaků)
vyselektují všechna slova z~rejstříku, která začínají zadaným
podřetězcem.

V~okamžiku, kdy je vybráno vstupní slovo, je tento řetězec zaslán do
služby \emph{analyze} (2), která řeší všechny záležitosti týkající se
derivační sítě DeriNet a~překladu. V~této službě se při jejím volání
inicializuje objekt, který bude cílovým výstupem této služby -- tento
objekt nazýváme \emph{infoBase} a~skládá se z~několika atributů:

\begin{itemize}
\tightlist
\item
  vstup v~českém jazyce (při inicializaci je za jeho hodnotu přiřazeno
  vstupní slovo);
\item
  vstup v~anglickém jazyce (při inicializaci je spuštěn automatický
  překlad);
\item
  základové slovo v~českém jazyce (tento a~zbytek atributů zůstávají při
  inicializaci prázdné);
\item
  základové slovo v~anglickém jazyce;
\item
  slovotvorný typ;
\item
  prefigovanost;
\item
  prefix;
\item
  derivační proces;
\item
  rod;
\item
  derivační cesta.
\end{itemize}

Při inicializaci se souběžně načítají data z~interního uložiště, která
jsou následně zkonvertována do použitelného datového formátu (3, 4, 5),
konkrétně jde o~databázi DeriNet, česko-anglický slovník
Glosbe\footnote{https://glosbe.com/ -- jedná se o~open-source projekt pod licencí CC-BY-SA z~něhož byla vyextrahována data pro naše použití.}
a~seznam výjimek (viz kapitola
\ref{zpracovanuxe9-slovotvornuxe9-sufixy}).

Dalším krokem je postupné procházení derivační sítě DeriNet, z~níž
potřebujeme vyextrahovat výše zmíněné lingvistické informace, procházíme
tedy derivační řetězec směrem od slova vstupního k~jeho slovu
základovému do té chvíle, než se zamění slovní druh, pak přistupujeme
k~volání funkce \emph{checkDertivationType} (6). Ta nám porovná řetězce
obou slov a~určí, o~jaký se jedná derivační proces a~slovotvorný typ.
Dále určuje algoritmus z~celého derivačního řetězce prefigovanost
(popřípadě zjišťuje o~jaký prefix se přesně jedná), zaznamená si rod
vstupního slova (ve výjimkách máme tedy v~případě slovotvorného sufixu
\emph{-tel} taková slova, která jsou neživotnými maskuliny) a~ukládá si
základové slovo spolu s~jeho anglickým ekvivalentem (u~českých
odvozených slov ve slovníku často překlad chybí). Posléze služba
\emph{analyze} vrací již vyplněný objekt \emph{infoBase} zpátky do
komponenty \emph{insert-word}, vrácený objekt může vypadat například
takto:

\begin{verbatim}
1.  czechInput:  "zpracovatel"
2.  czechParent:  "zpracovat"
3.  derProcess:  "suffixation"
4.  derivType:  "tel"
5.  derivationPath:  (2) [{…},  {…}] // derivační řetězec
6.  englishInput:  ""
7.  englishParent:  "processes"
8.  gender:  "M"
9.  isPrefig:  true
10. prefix:  "z"
\end{verbatim}

Nyní se dostáváme do kroku č. 8, kdy je objekt \emph{infoBase} předán
druhé službě s~názvem \emph{createDefiniton}, jejímž účelem je vytvořit
samotné slovníkové heslo do formy objektu \emph{definiton}. Jednotlivé
části jsou zpracovány separátně, a~to jednak z~důvodu přehlednosti kódu,
ale taktéž z~ryze praktické příčiny, tedy aby s~výsledným objektem byla
co nejjednodušší manipulace na úrovní HTML šablony.

Na začátku inicializujeme obecnou slovotvornou definici podle rodu ze
vstupního objektu \emph{infoBase} a~poté přistupujeme k~funkci
\emph{getThirdPerson} (9), v~rámci které vytváříme slovesnou formu ve
třetí osobě podle lingvistických pravidel popsaných v~kapitole
\ref{slovotvornuxe1-definice}.

Následně dotvoříme derivační a~morfologickou informaci z~poznatků
získaných ze služby \emph{analyze} a~vracíme objekt \emph{definiton}
zpět do komponenty \emph{insert-word} (10). Pokud je vrácen objekt
správného typu, je HTML šabloně umožněno skrze direktivy frameworku
Angular přistoupit k~jednotlivým častém definice a~vykreslit je do
označených míst v~šabloně (11).

Výhoda tohoto objektového přístupu je taková, že se na jednotlivých
jasně otypovaných místech v~kódu provádí pouze jeden specifický úkol.
Tím je do určité míry zajištěna robustnost celkové aplikace, u~níž je
při výpadku jedné služby/funkce jednoduché lokalizovat místo chyby.
