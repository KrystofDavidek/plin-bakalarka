\part{Praktická část}

\hypertarget{zpracovanuxe9-slovotvornuxe9-sufixy}{%
\chapter{Zpracované slovotvorné
sufixy}\label{zpracovanuxe9-slovotvornuxe9-sufixy}}

Cílem bakalářské práce bylo vytvořit elektronický slovník s~definicemi
založenými na derivačních rysech slovotvorně motivovaných slov ve formě
mobilní aplikace, proto si v~této kapitole praktické části popíšeme,
jaké slovotvorné sufixy byly zpracovány a~jakým způsobem probíhá proces
tvoření slovotvorných definic. V~druhé kapitole si pak rozebereme
technickou část spolu s~použitými technologiemi a~popisem implementace
samotné mobilní aplikace.

Při výběru slovotvorných typů ke zpracování byl brán zřetel na jejich
frekvenci a~produktivitu ve spojitosti s~derivací u~substantiv --
takovým nejvýraznějším typem jsou právě substantiva označující názvy
živých bytostí podle jejich činností zakončená na sufix \emph{-tel},
a~ta spadají do slovotvorné třídy činitelských jmen.
\parencite[17]{dokulil67}

\hypertarget{slovotvornuxfd-typ--tel}{%
\section{Slovotvorný typ -tel}\label{slovotvornuxfd-typ--tel}}

Pro naše účely bylo důležité vybrat takový slovotvorný typ, u~něhož se
převážně shoduje slovotvorný a~lexikální význam. Tato podmínka je
splněna, protože z~výzkumné práce Adriany Válkové (To jsem šplhoun, co?
XXX) vyplývá, že z~1 129 zkoumaných lemmat zakončených příponou
\emph{-tel} pouze u~3,03~\% z~nich lexikální význam nahrazuje význam
strukturní (to se týká převážně neživotných substantiv typu
\emph{jmenovatel}, \emph{dělitel} atd.) a~celkově je u~4,42~\% lexikální
význam obecnější
(např.\emph{vnímatel}\footnote{význam slova *vnímatel* je dle SSJČ „kdo uvědoměle vnímá umělecké dílo“~\parencite{ssjc}, zde došlo prokazatelně k~lexikalizaci slovotvorného významu.}).
Tento lingvistický výzkum byl založen na datech z~korpusu SYNv6
a~jednotlivé typy významů byly porovnávány prostřednictvím výkladových
slovníků. citace\{Adri-XXX\}

Jak bylo již naznačeno, obecný význam činitelských jmen je podle
Dokulila „názvy osob a~živých bytostí vůbec podle povahy a~druhu jejich
činností``~\parencite[17]{dokulil67}. Sufix \emph{-tel} tak vyjadřuje,
že takto odvozený pojem je subjektem děje základového slovesa, s~tím že
nejčastěji jde o~aktivní\footnote{To neplatí u~substantiv *trpitel*, *truchlitel* a~*bydlitel*, které jsou odvozený ze stavových sloves.~\parencite[17]{dokulil67}}
účast subjektu na ději (např. subjekt označen slovem \emph{učitel}
vykonává takovou činnost, kterou vyjadřuje sloveso \emph{učit}, z~něhož
je výraz odvozený). U~tohoto slovotvorného typu se typicky jedná
o~mužská substantiva, nicméně se najdou i~výjimky v~podobě neživotných
substantiv (viz předchozí odstavec).~\parencite{simandl2016}

Nejčastěji jsou substantiva tohoto slovotvorného typu odvozena od
imperfektiv, ze zkoumaných 1129 lemmat je jich takto derivováno
přibližně 74,3~\%, navíc se určitá část substantiv vzniklých z~perfektiv
chová, jako byly odvozeny z~imperfektiv, jde typicky o~názvy profesí
(\emph{zastoupit} --\textgreater{} \emph{zastupitel}) a~názvy osob, pro
které je daná činnost typická, ale nejsou označovány za samostatné
profese. (\emph{chovat} --\textgreater{} \emph{chovatel}).
citace\{Adri-XXX\}
