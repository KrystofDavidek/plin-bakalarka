\part{Praktická část}

\hypertarget{zpracovanuxe9-slovotvornuxe9-sufixy}{%
\chapter{Zpracované slovotvorné
sufixy}\label{zpracovanuxe9-slovotvornuxe9-sufixy}}

Cílem bakalářské práce bylo vytvořit elektronický slovník s~definicemi
založenými na derivačních rysech slovotvorně motivovaných slov ve formě
mobilní aplikace, proto si v~této kapitole praktické části popíšeme,
jaké slovotvorné sufixy byly zpracovány a~jakým způsobem probíhá proces
tvoření slovotvorných definic. V~druhé kapitole si pak rozebereme
technickou část spolu s~použitými technologiemi a~popisem implementace
samotné mobilní aplikace.

Při výběru slovotvorných typů ke zpracování byl brán zřetel na jejich
frekvenci a~produktivitu ve spojitosti s~derivací u~substantiv --
takovým nejvýraznějším typem jsou substantiva označující názvy živých
bytostí podle jejich činností zakončená na sufix \emph{-tel}, a~ta
spadají do slovotvorné třídy činitelských jmen.
\parencite[17]{dokulil67} V~této fázi je v~předkládané aplikaci
zpracován právě tento slovotvorný typ i~s~jeho rodovou alternativou
\emph{-telka}, jež má obdobné charakteristiky.

\hypertarget{slovotvornuxfd-typ--tel}{%
\section{Slovotvorný typ -tel}\label{slovotvornuxfd-typ--tel}}

\hypertarget{charakteristika}{%
\subsection{Charakteristika}\label{charakteristika}}

Pro naše účely bylo důležité vybrat takový slovotvorný typ, u~něhož se
převážně shoduje slovotvorný a~lexikální význam. Tato podmínka je zde
splněna, protože z~výzkumné práce Adriany Válkové (To jsem šplhoun, co?
XXX) vyplývá, že z~1 129 zkoumaných lemmat zakončených příponou
\emph{-tel} pouze u~3,03~\% z~nich lexikální význam nahrazuje význam
strukturní (to se týká převážně neživotných substantiv typu
\emph{jmenovatel}, \emph{dělitel} atd.) a~celkově je u~4,42~\% lexikální
význam obecnější
(např.\emph{vnímatel}\footnote{Význam slova vnímatel je dle SSJČ~\parencite{ssjc} „kdo uvědoměle vnímá umělecké dílo“, zde došlo prokazatelně k~lexikalizaci slovotvorného významu.}).
Tento lingvistický výzkum byl založen na datech z~korpusu SYNv6
a~jednotlivé typy významů byly porovnávány prostřednictvím výkladových
slovníků. citace\{Adri-XXX\}

Jak bylo již naznačeno, obecný význam činitelských jmen je podle
Dokulila definován jako „názvy osob a~živých bytostí vůbec podle povahy
a~druhu jejich činností``~\parencite[17]{dokulil67}. Sufix \emph{-tel}
tak vyjadřuje, že takto odvozený pojem je subjektem děje základového
slovesa, s~tím že nejčastěji jde
o~aktivní\footnote{To neplatí u~substantiv trpitel, truchlitel a~bydlitel, která jsou odvozena ze stavových sloves.~\parencite[17]{dokulil67}}
účast subjektu na ději (např. subjekt označen slovem \emph{učitel}
vykonává takovou činnost, kterou vyjadřuje sloveso \emph{učit}, z~něhož
je výraz odvozený). U~tohoto slovotvorného typu se typicky jedná
o~mužská životná substantiva, nicméně se najdou i~výjimky v~podobě
neživotných substantiv (viz předchozí odstavec).~\parencite{simandl2016}

Nejčastěji jsou substantiva tohoto slovotvorného typu odvozena od sloves
nedokonavých (imperfektiv), ze zkoumaných 1129 lemmat je jich takto
derivováno přibližně 74,3~\%, navíc se určitá část substantiv vzniklých
ze sloves dokonavých (perfektiv) chová tak, jako by byla odvozena
z~imperfektiv, jde typicky o~názvy profesí (\emph{zastoupit}
--\textgreater{} \emph{zastupitel}) a~názvy osob, pro které je daná
činnost typická, ale nejsou označovány za samostatné profese.
(\emph{chovat} --\textgreater{} \emph{chovatel}). citace\{Adri-XXX\}
Nicméně je zapotřebí poznamenat, že některá substantiva se mohou objevit
v~obou vidových podobách (např. \emph{pojistitel} a~\emph{pojišťovatel}
nebo \emph{zhotovitel} a~\emph{zhotovovatel}).~\parencite{simandl2016}

\hypertarget{slovotvornuxe1-definice}{%
\subsection{Slovotvorná definice}\label{slovotvornuxe1-definice}}

Heslo derivačního slovníku se skládá ze tří částí, a~to z~definice
v~obecném slova smyslu, derivační informace a~morfologické informace.
Definice je tvořena částečnou slovotvornou analýzou (segmentace slova na
slovotvornou bázi a~formant/y) a~samotnou slovotvornou definicí
založenou na strukturním významu zadaného slova (ta je i~v~anglickém
jazyce).

Slovotvorná definice je tvořena dvěma kroky, první fází je určení
slovesného vidu fundujícího/motivující slovesa. Ten lze do určité míry
vyextrahovat z~derivační sítě DeriNet -- a~to analýzou slovotvorného
řetězce:

\begin{itemize}
\tightlist
\item
  sloveso je perfektum, pokud řetězec obsahuje tutéž slovesnou formu
v~neprefigované podobě ve dvou předchozích slovotvorných krocích, mějme
  například řetězec \emph{pracovat} (2. krok) --\textgreater{}
  \emph{zpracovat} (1. krok) --\textgreater{} \emph{zpracovatel} --
  slovotvorná definice tak bude znít „ten, kdo zpracoval nebo
  zpracuje``;
\item
  sloveso je sekundární imperfektivum, pokud řetězec obsahuje rozdíl ve
  dvou předchozích slovotvorných krocích v~rámci specifického sufixu
u~jedné slovesné formy, příkladem může být řetězec \emph{pracovat}
  --\textgreater{} \emph{zpracovat} (2. krok) --\textgreater{}
  \emph{zpracovávat} (1. krok) --\textgreater{} \emph{zpracovávatel} --
  slovotvorná definice tak bude znít „ten, kdo zpracovává``;
\item
  sloveso je imperfektivum, pokud není perfektum a~zároveň není
  sekundární imperfektivum, například \emph{myslit} --\textgreater{}
  \emph{myslitel} -- slovotvorná definice bude znít „ten, kdo myslí``.
\end{itemize}

Druhá fáze je tvorba samotné slovotvorné definice napříč různými
slovesnými třídami, pro potřeby automatického zpracování jsme vytvářeli
definice na základě vlastních podvzorů, které jsou popsány určitým
regulárním výrazem viz tabulka níže.

The table \ref{table:1} is an example of referenced \LaTeX elements.

\begin{table}[h!]
\centering
\begin{tabular}{||c c c c||} 
 \hline
 Col1 & Col2 & Col2 & Col3 \\ [0.5ex] 
 \hline\hline
 1 & 6 & 87837 & 787 \\ 
 2 & 7 & 78 & 5415 \\
 3 & 545 & 778 & 7507 \\
 4 & 545 & 18744 & 7560 \\
 5 & 88 & 788 & 6344 \\ [1ex] 
 \hline
\end{tabular}
\caption{Table to test captions and labels}
\label{table:1}
\end{table}
