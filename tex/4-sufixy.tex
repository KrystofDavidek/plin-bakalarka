\part{Praktická část}

\hypertarget{zpracovanuxe9-slovotvornuxe9-sufixy}{%
\chapter{Zpracované slovotvorné
sufixy}\label{zpracovanuxe9-slovotvornuxe9-sufixy}}

Cílem bakalářské práce bylo vytvořit elektronický slovník s~definicemi
založenými na derivačních rysech slovotvorně motivovaných slov ve formě
mobilní aplikace, proto si v~této kapitole praktické části popíšeme,
jaké slovotvorné sufixy byly zpracovány a~jakým způsobem probíhá proces
tvoření slovotvorných definic. V~druhé kapitole si pak rozebereme
technickou část spolu s~návrhem a~implementací samotné mobilní aplikace.

Při výběru slovotvorných typů ke zpracování byl brán zřetel na jejich
frekvenci a~produktivitu ve spojitosti s~derivací u~substantiv --
takovým nejvýraznějším typem jsou substantiva označující názvy živých
bytostí podle jejich činností zakončená na sufix \emph{-tel}, a~ta
spadají do slovotvorné třídy činitelských jmen.
\parencite[17]{dokulil67} V~této fázi je v~předkládané aplikaci
zpracován právě tento slovotvorný typ i~s~jeho rodovou alternativou
\emph{-telka}, jež má obdobné charakteristiky.

\hypertarget{slovotvornuxfd-typ--tel}{%
\section{Slovotvorný typ -tel}\label{slovotvornuxfd-typ--tel}}

\hypertarget{charakteristika}{%
\subsection{Charakteristika}\label{charakteristika}}

Pro účely této bylo důležité vybrat takový slovotvorný typ, u~něhož se
převážně shoduje slovotvorný a~lexikální význam. Tato podmínka je zde
splněna, protože z~1 129 zkoumaných lemmat zakončených sufixem
\emph{-tel} pouze u~3,03~\% z~nich strukturní význam neodpovídá významu
lexikálnímu (to se týká převážně neživotných substantiv typu
\emph{jmenovatel}, \emph{dělitel} atd.) a~celkově je u~4,42~\%
strukturní význam obecnější (např.
\emph{vnímatel}\footnote{Význam slova vnímatel je dle SSJČ „kdo uvědoměle vnímá umělecké dílo“~\parencite{ssjc}, došlo tedy k~zúžení lexikálního významu.}).
Tento lingvistický výzkum byl založen na datech z~korpusu SYNv6
a~jednotlivé typy významů byly porovnávány prostřednictvím výkladových
slovníků.~\parencite{adri}

Jak bylo již naznačeno, obecný význam činitelských jmen je podle
Dokulila definován jako „názvy osob a~živých bytostí vůbec podle povahy
a~druhu jejich činností``~\parencite[17]{dokulil67}. Sufix \emph{-tel}
tak vyjadřuje, že takto odvozený pojem je subjektem děje základového
slovesa s~tím, že nejčastěji jde
o~aktivní\footnote{To neplatí u~substantiv trpitel, truchlitel a~bydlitel, která jsou odvozena ze stavových sloves.~\parencite[17]{dokulil67}}
účast subjektu na ději (např. subjekt označen slovem \emph{učitel}
vykonává takovou činnost, kterou vyjadřuje sloveso \emph{učit}, z~něhož
je výraz odvozený). U~tohoto slovotvorného typu se typicky jedná
o~mužská životná substantiva, nicméně se najdou i~výjimky v~podobě
neživotných substantiv (viz výše).~\parencite{simandl2016}

Nejčastěji jsou substantiva tohoto slovotvorného typu odvozena od sloves
nedokonavých (imperfektiv), ze zkoumaných 1129 lemmat je jich takto
derivováno přibližně 74,3~\%, navíc se určitá část substantiv vzniklých
ze sloves dokonavých (perfektiv) chová tak, jako by byla odvozena
z~imperfektiv. Jde typicky o~názvy profesí (\emph{zastoupit}
$\rightarrow$ \emph{zastupitel}) a~názvy osob, pro které je daná
činnost typická, ale nejsou označovány za samostatné profese
(\emph{chovat} $\rightarrow$ \emph{chovatel}). Navíc je možné
sledovat progresivní tendenci produktivní V. slovesné třídy, v~níž
vznikají vedle tvarů utvořených z~perfektivních sloves
(\emph{zhotovitel}) i~tvary ze sloves imperfektivních
(\emph{zhotovovatel}).~\parencite{adri}

\hypertarget{slovnuxedkovuxe9-heslo}{%
\subsection{Slovníkové heslo}\label{slovnuxedkovuxe9-heslo}}

Heslo derivačního slovníku se skládá ze tří částí:

\begin{itemize}
\tightlist
\item
  z~definice v~obecném slova smyslu;
\item
  derivační informace;
\item
  morfologické informace.
\end{itemize}

Definice je tvořena částečnou slovotvornou analýzou (segmentace slova na
slovotvornou bázi a~formant/y, například \emph{učitel} $\rightarrow$
\emph{učit-tel}) a~samotnou slovotvornou definicí založenou na
strukturním významu zadaného slova.

\hypertarget{slovotvornuxe1-definice}{%
\subsubsection{Slovotvorná definice}\label{slovotvornuxe1-definice}}

Slovotvorná definice je utvářena prostřednictvím dvou po sobě jdoucích
kroků -- v~prvním kroku dochází k~odhalení verbálního aspektu
fundujícího/motivující slovesa a~ve druhém se pak vstupnímu řetězci
přiřazuje jedna z~předem vytvořených definic.

\hypertarget{prvni-faze}{%
\subsubsection*{První fáze -- odhalení slovesného vidu}\label{prvni-faze}}

Verbální aspekt lze do určité míry vyextrahovat z~derivační sítě DeriNet
-- a~to analýzou slovotvorného řetězce:

\begin{itemize}
\tightlist
\item
  sloveso je perfektum: řetězec obsahuje tutéž slovesnou formu
v~neprefigované podobě ve dvou předchozích slovotvorných krocích. Mějme
  například řetězec \emph{pracovat} (2. krok) $\rightarrow$
  \emph{\textbf{z}pracovat} (1. krok) $\rightarrow$
  \emph{zpracovatel} -- slovotvorná definice tak bude znít „ten, kdo
  zpracoval nebo zpracuje``;
\item
  sloveso je sekundární imperfektivum: řetězec obsahuje rozdíl ve dvou
  předchozích slovotvorných krocích v~rámci specifického sufixu u~jedné
  slovesné formy, příkladem může být řetězec \emph{pracovat}
  $\rightarrow$ \emph{zpracovat} (2. krok) $\rightarrow$
  \emph{zpraco\textbf{vá}vat} (1. krok) $\rightarrow$
  \emph{zpracovávatel} -- slovotvorná definice tak bude znít „ten, kdo
  zpracovává``;
\item
  sloveso je imperfektivum: pokud není perfektum a~zároveň není
  sekundární imperfektivum, například \emph{myslit} $\rightarrow$
  \emph{myslitel} -- slovotvorná definice je „ten, kdo myslí``.
\end{itemize}

\hypertarget{druha-faze}{%
\subsubsection*{Druhá fáze -- přiřazení definice}\label{druha-faze}}

Pro potřeby automatického zpracování jsme vytvářeli definice na základě
vlastních podvzorů, které jsou popsány určitými regulárními výrazy, viz
následující pravidla (dubletní varianty jsou označeny v~kulatých
závorkách):

\begin{itemize}
\tightlist
\item
  {[}\^{}ch{]}*ovatel\footnote{V tomto případě existují výjimky typu kl?ovatel, kdy zní definice u~neprefigovaného slovesa „ten, kdo kl?ove (kl?ová)“ a~u~prefigovaného „ten, kdo .*kl?oval nebo .*kl?ove (.*kl?ová). Mezi tato slova patří .*snovatel, .*plovatel, .*kovatel a~.*klovatel.}
  $\rightarrow$ Je prefigované?

  \begin{itemize}
  \tightlist
  \item
    ne $\rightarrow$ „ten, kdo .*uje``
  \item
    ano

    \begin{itemize}
    \tightlist
    \item
      Existuje ve slovotvorném řetězci sloveso ve tvaru .*ovat
      $\rightarrow$ „ten, kdo .*oval nebo .*uje``
    \item
      Existuje ve slovotvorném řetězci sloveso ve tvaru .*it nebo .*nout
      nebo .*{[}aá{]}t? $\rightarrow$ „ten, kdo .*uje``
    \end{itemize}
  \end{itemize}
\item
  .*chovatel $\rightarrow$ Je prefigované?

  \begin{itemize}
  \tightlist
  \item
    ne $\rightarrow$ „ten, kdo .*á``
  \item
    ano $\rightarrow$ „ten, kdo .*ten, kdo .*al nebo .*á``
  \end{itemize}
\item
  {[}\^{}o{]}*{[}iíyýaá{]}vatel $\rightarrow$ „ten, kdo
  .*{[}íýá{]}vá``
\item
  {[}\^{}o{]}*ěvatel $\rightarrow$ „ten, kdo .*ívá``
\item
  .*itel $\rightarrow$ Je prefigované?

  \begin{itemize}
  \tightlist
  \item
    ne

    \begin{itemize}
    \tightlist
    \item
      Je sloveso ve tvaru .*u.it? $\rightarrow$ „ten, kdo .*u.í``
    \item
      Je sloveso ve tvaru {[}\^{}u{]}*{[}eěi{]}t? $\rightarrow$ „ten,
      kdo .*í``
    \item
      Je sloveso ve tvaru .*u.ovat a~existuje zároveň ve slovotvorném
      řetězci sloveso ve tvaru .*ou.it? $\rightarrow$ „ten, kdo
      .*ou.il nebo .*ou.í``
    \end{itemize}
  \item
    ano

    \begin{itemize}
    \tightlist
    \item
      Existuje ve slovotvorném řetězci sloveso ve tvaru .*ou.it?
      $\rightarrow$ „ten, kdo .*ou.il nebo .*ou.í``
    \item
      Existuje ve slovotvorném řetězci sloveso ve tvaru .*ovat a~zároveň
      v~řetězci neexistuje sloveso ve tvaru .*it? $\rightarrow$ „ten,
      kdo .*oval``
    \item
      Je sloveso ve tvaru .*{[}eě{]}t? $\rightarrow$ „ten, kdo
      .*{[}eě{]}l nebo .*í``
    \item
      Je sloveso ve tvaru .*{[}\^{}ou{]}.*it $\rightarrow$ „ten, kdo
      .*il nebo .*í``
    \end{itemize}
  \end{itemize}
\item
  {[}\^{}zb{]}*atel\footnote{Zde do výjimek u~neprefigovaných podob spadají slova .*zobatel, .*hýbatel, .*kazatel a~.*tazatel.}
  $\rightarrow$ Je prefigované?

  \begin{itemize}
  \tightlist
  \item
    ne $\rightarrow$ „ten, kdo .*á``
  \item
    ano $\rightarrow$ „ten, kdo .*al nebo .*á``
  \end{itemize}
\item
  .*zatel $\rightarrow$ Je prefigované?

  \begin{itemize}
  \tightlist
  \item
    ne $\rightarrow$ „ten, kdo .*že``
  \item
    ano $\rightarrow$ ten, kdo .*zal nebo .*že``
  \end{itemize}
\item
  .*batel $\rightarrow$ Je prefigované?

  \begin{itemize}
  \tightlist
  \item
    ne $\rightarrow$ „ten, kdo .*bá`` (.*be)
  \item
    ano $\rightarrow$ „ten, kdo .*bal nebo .*bá`` (.*be)
  \end{itemize}
\item
  p{[}ií{]}satel $\rightarrow$ Je prefigované?

  \begin{itemize}
  \tightlist
  \item
    ne $\rightarrow$ „ten, kdo píše``
  \end{itemize}
\end{itemize}

Veškerá výše zmíněná pravidla jsou taktéž aplikovatelná na slovotvorný
typ \emph{-telka}, pro kterou je slovotvorná definice v~obecné rovině:
„ta, která .*``. Pokud budeme mít například za vstup slovo
\emph{vybudovatelka}, tak jeho definice bude znít „ta, která vybudovala
nebo vybuduje (vybudovat)``. Za definicí samotnou je v~závorce uvedeno
základové sloveso v~infinitivním tvaru, z~něhož bylo činitelské jméno
derivováno.

Slovník vytváří slovotvorné definice i~v~anglickém jazyce, v~rámci
kterého je definice zobecněná na „someone who .*`` s~tím že je na konci
v~závorce specifikováno, o~jaký se jedná jedná rod, příkladem může být
znovu výraz \emph{vybudovatelka} s~anglickou definicí „someone who
builds (feminine)``.

\hypertarget{derivaux10dnuxed-a-morfologickuxe1-informace}{%
\subsubsection{Derivační a~morfologická
informace}\label{derivaux10dnuxed-a-morfologickuxe1-informace}}

Kromě hlavní definice slovníkové heslo obsahuje ještě dodatečnou
derivační a~morfologickou informaci. První z~nich obsahuje základové
slovo, z~něhož byl vstupní výraz odvozen, a~typ derivačního procesu --
v~případě tohoto slovotvorného typu se jedná o~sufixaci.

Morfologická informace se pak skládá ze slovního druhu vstupního výrazu,
z~jeho rodu a~z~určitého morfologického paradigmatu, do něhož bylo slovo
zařazeno.
