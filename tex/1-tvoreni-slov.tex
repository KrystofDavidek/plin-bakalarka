\hypertarget{slovotvorba}{%
\chapter{Slovotvorba}\label{slovotvorba}}

V~této úvodní kapitole si představíme slovotvorbu tak, jak ji vnímá
dnešní tuzemská lingvistika, a~ve stručnosti si popíšeme její historický
vývoj v~českém prostředí.

\hypertarget{definice}{%
\section{Definice}\label{definice}}

Za slovotvorbu (derivologii) lze v~lingvistice v~nejobecnějším pojetí
považovat nauku o~tvoření slov, jde tedy o~takovou vědní disciplínu,
která zkoumá a~popisuje procesy, které doprovází vznik nových slov
v~daném jazyce.

V~užším pojetí pod tvořením slov myslíme právě ty procesy, jež pod sebou
zahrnují různé způsoby, postupy a~prostředky, díky nimž dochází
k~produkci nových slov, tedy neologismů. (Tyto slovotvorné postupy jsou
často reprodukovány.) Důležitou součástí slovotvorby jsou pak výsledné
stavy těchto procesů, které označujeme jako slovotvornou stavbu.
\parentcite[92]{cechova00}

Ze slovotvorné stavby lze typicky odvodit samotnou genezi daného
neologismu, to znamená, že by mělo být zřejmé, z~jakého slova je slovo
nové vytvořeno (například u~slova \emph{ne-dobrý} lze lehko vyčíst, že
je pomocí prefixu \emph{ne} odvozeno od adjektiva \emph{dobrý}). To ale
nemusí platit vždy, často není u~odvozovaných slov (viz níže) úplně
zřejmé, které slovo bylo tvarem základovým.
\parentcite[92--93]{cechova00}
