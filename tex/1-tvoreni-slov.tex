\hypertarget{slovotvorba}{%
\chapter{Slovotvorba}\label{slovotvorba}}

V~této úvodní kapitole si představíme slovotvorbu tak, jak ji vnímá
dnešní tuzemská lingvistika, a~ve stručnosti si popíšeme hlavní přístupy
k~této problematice v~českém prostředí.

\hypertarget{uvedenuxed-do-problematiky}{%
\section{Uvedení do problematiky}\label{uvedenuxed-do-problematiky}}

Za slovotvorbu (derivologii) lze v~lingvistice v~nejobecnějším pojetí
považovat nauku o~tvoření slov, jde tedy o~takovou vědní disciplínu,
která zkoumá a~popisuje procesy, které doprovází vznik nových slov
v~daném jazyce.

V~užším pojetí pod tvořením slov myslíme právě ty procesy, jež pod sebou
zahrnují různé způsoby, postupy a~prostředky, díky nimž dochází
k~produkci nových slov, tedy neologismů. (Tyto slovotvorné postupy jsou
často reprodukovány.) Důležitou součástí slovotvorby jsou pak výsledné
stavy těchto procesů, které označujeme jako slovotvornou stavbu.
\parentcite[92]{cechova00}

Ze slovotvorné stavby lze typicky odvodit samotnou genezi daného
neologismu, to znamená, že by mělo být zřejmé, z~jakého slova je slovo
nové vytvořeno (například u~slova \emph{ne-dobrý} lze lehko vyčíst, že
je pomocí prefixu \emph{ne} odvozeno od adjektiva \emph{dobrý}). To ale
nemusí platit vždy, často není u~odvozovaných slov (viz kapitola NEZNÁM
JEŠTĚ ČÍSLO) úplně zřejmé, které slovo bylo tvarem základovým.
\parentcite[92--93]{cechova00}

\hypertarget{pux159uxedstupy-a-teorie}{%
\section{Přístupy a~teorie}\label{pux159uxedstupy-a-teorie}}

Jako většina lingvistických disciplín, prošla si i~tato oblast svým
vlastním vývojem, jehož dobře mapuje kapitola \emph{Slovotvorba}
v~publikaci Zdeny Rousínové -- \emph{Dějiny českého jazykovědného
myšlení}. Pro účely této práce vynecháme historický exkurz a~zaměříme se
na nejvýznamnější synchronní přístupy v~rámci tvoření slov ve 20. a~21.
století.

Za jedno z~prvních významných děl lze považovat \emph{Mluvnici spisovné
češtiny} od Františka Trávníčka (1. vydání vyšlo v~roce 1948), která
jako jedna z~prvních gramatik synchronně popisuje problematiku
slovotvorby. Zde je nutno podotknout, že zde Trávníček tvoření nových
slov úzce spojoval s~lexikologií a~nevyčlenil jej tak jako samostatnou
disciplínu.~\parencite[263]{rousinova}

Dalším výrazným příspěvkem do teorie slovotvorby přispěl Vladimír
Šmilauer svojí publikací \emph{Novočeské tvoření slov}, které bylo
napsáno v~letech 1937--1938, nicméně z~geopolitických důvodů dílo vyšlo
až v~roce 1971. Šmilauer se zde primárně zaměřil na popis procesu,
kterým si procházejí mluvčí, chtějí-li vytvářet nové pojmenování pro
nějaký jev.~\parencite[265]{rousinova}

Za přelomové dílo, které proměnilo českou slovotvorbu do dnešních podob,
je publikace Miloše Dokulila \emph{Teorie odvozování} (vyšla v~roce
1962) -- tato práce byla první částí kolektivního díla \emph{Tvoření
slov v~češtině}, na níž se podílel kolektiv pracovníků Ústavu pro jazyk
český Československé akademie věd. Cílem tohoto textu bylo podat
komplexní metodologický základ, jenž měl být svoji univerzálností platný
i~na další slovanské jazyky.~\parencite[267]{rousinova} Tato práce dala
za vznik všeobecně přijímané onomaziologické teorii slovotvorby, kterou
si důsledněji rozebereme v~druhé kapitole, nicméně je zde vhodné
poznamenat, že na Dokulilovo dílo bylo navázáno dalšími pracemi, které
jeho teorii dále rozvíjeli, aniž by jeho přístup ztratil na platnosti.
I~díky tomu se do teď tato teorie považuje za nejúplnější analýzu
slovotvorné struktury češtiny.~\parencite[272--273]{rousinova}
