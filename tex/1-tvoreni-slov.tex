\hypertarget{slovotvorba}{%
\chapter{Slovotvorba}\label{slovotvorba}}

V~této úvodní kapitole si představíme slovotvorbu tak, jak ji vnímá
dnešní lingvistika, a~ve stručnosti si popíšeme její historický vývoj.

\hypertarget{definice}{%
\section{Definice}\label{definice}}

Za slovotvorbu (derivologii) lze v~lingvistice v~nejobecnějším pojetí
považovat nauku o~tvoření slov, jde tedy o~takovou vědní disciplínu,
která zkoumá a~popisuje procesy, které doprovází vznik nových slov
v~daném jazyce.

V~užším pojetí pod tvořením slov myslíme právě ony procesy, jež pod
sebou zahrnují různé způsoby, postupy a~prostředky, díky nimž dochází
k~produkci nových slov, tedy neologismů. (Tyto slovotvorné postupy jsou
často reprodukovány.) /parentcite{[}92{]}\{cechova00\} 
