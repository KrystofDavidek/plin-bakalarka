\hypertarget{onomaziologickuxe1-teorie-slovotvorby}{%
\chapter{Onomaziologická teorie
slovotvorby}\label{onomaziologickuxe1-teorie-slovotvorby}}

\hypertarget{ukotvenuxed-teorie-struux10dnuxe1-historie}{%
\section{Ukotvení teorie, stručná
historie}\label{ukotvenuxed-teorie-struux10dnuxe1-historie}}

Jak bylo v~minulé kapitole naznačeno, onomaziologická teorie slovotvorby
(dále jen OTS) Miloše Dokulila výrazně ovlivnila přístup ke slovotvorbě
jako takové, protože si kladla za cíl popsat proces tvoření slov soudobé
češtiny a~zároveň definovat hranici s~dalšími lingvistickými
disciplínami jako jsou morfologie či lexikologie -- jedná se tedy o~ryze
synchronní přístup, který je založen na onomaziologii (obecné teorii
pojmenovávání).~\parencite{enc-ots17}

Onomaziologie zkoumá takové motivace, postupy a~prostředky,
prostřednictvím kterých jsou v~daném jazyce vyjadřovány určité obsahy.
Na rozdíl od sémaziologie postupuje od obsahu k~formě, to znamená, že
vychází od pojmu k~jazykovým znaků, kterými je daný pojem vyjádřen.
\parencite{enc-onomaz17}

Po metodologické stránce Dokulil u~tvoření slov rozlišuje aspekt
genetický, který je zaměřený na vlastní slovotvorné procesy, a~na aspekt
funkčně strukturní, jenž sleduje výsledek těchto procesů.
\parencite[9]{dokulil62}

\hypertarget{zpux16fsoby-a-prostux159edky-tvoux159enuxed-novuxfdch-pojmenovuxe1nuxed}{%
\section{Způsoby a~prostředky tvoření nových
pojmenování}\label{zpux16fsoby-a-prostux159edky-tvoux159enuxed-novuxfdch-pojmenovuxe1nuxed}}

\hypertarget{lexikuxe1lnuxed-a-strukturuxe1lnuxed-vuxfdznam}{%
\section{Lexikální a~strukturální
význam}\label{lexikuxe1lnuxed-a-strukturuxe1lnuxed-vuxfdznam}}

\hypertarget{slovotvornuxe9-vztahy}{%
\section{Slovotvorné vztahy}\label{slovotvornuxe9-vztahy}}

\hypertarget{fundace}{%
\subsection{Fundace}\label{fundace}}

\hypertarget{motivace}{%
\subsection{Motivace}\label{motivace}}

\hypertarget{onomaziologickuxe9-kategorie-a-jejich-klasifikace}{%
\section{Onomaziologické kategorie a~jejich
klasifikace}\label{onomaziologickuxe9-kategorie-a-jejich-klasifikace}}

\hypertarget{substance-vlastnost-dux11bj-okolnost}{%
\subsection{Substance, vlastnost, děj,
okolnost}\label{substance-vlastnost-dux11bj-okolnost}}

\hypertarget{mutace}{%
\subsection{Mutace}\label{mutace}}

\hypertarget{modifikace}{%
\subsection{Modifikace}\label{modifikace}}

\hypertarget{transpozice}{%
\subsection{Transpozice}\label{transpozice}}

\hypertarget{slovotvornuxe9-tux159uxeddy-a-typy}{%
\section{Slovotvorné třídy
a~typy}\label{slovotvornuxe9-tux159uxeddy-a-typy}}

\hypertarget{definice}{%
\subsection{Definice}\label{definice}}

\hypertarget{dux11blenuxed}{%
\subsection{Dělení}\label{dux11blenuxed}}
