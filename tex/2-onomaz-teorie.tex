\hypertarget{onomaziologickuxe1-teorie-slovotvorby}{%
\chapter{Onomaziologická teorie
slovotvorby}\label{onomaziologickuxe1-teorie-slovotvorby}}

Jak bylo v~minulé kapitole naznačeno, onomaziologická teorie slovotvorby
(dále jen OTS) je v~českém prostředí spjata s~dílem Miloše Dokulila --
\emph{Tvoření slov v~češtině} a~\emph{Tvoření slov v~češtině 2} (TSČ
a~TSČ 2). Jeho teorie výrazně ovlivnila přístup ke slovotvorbě jako
takové, protože si kladla za cíl popsat proces tvoření slov soudobé
češtiny a~zároveň definovat hranici s~dalšími lingvistickými
disciplínami jako jsou morfologie či lexikologie -- jedná se tedy o~ryze
synchronní přístup, který je založen na onomaziologii (obecné teorii
pojmenovávání).~\parencite{enc-ots17}

Onomaziologie zkoumá takové motivace, postupy a~prostředky,
prostřednictvím kterých jsou v~daném jazyce vyjadřovány určité obsahy.
Na rozdíl od sémaziologie postupuje od obsahu k~formě, to znamená, že
vychází od pojmu k~pojmenování, kterým je daný pojem vyjádřen.
\parencite{enc-onomaz17} Díky těmto termínům OTS terminologicky vyřešila
problematiku výrazů \emph{pojem} a~\emph{slovo}, která byla dříve často
nesystematicky zaměňována.~\parencite[267]{rousinova07}

Po metodické stránce Dokulil u~tvoření slov rozlišuje aspekt genetický,
který je zaměřený na vlastní slovotvorné procesy, a~na aspekt funkčně
strukturní, jenž sleduje výsledek těchto procesů (utvářenost slov).
Nicméně jak sám autor teorie potvrzuje, je mezi těmito přístupy těsná
spojitost.~\parencite[9]{dokulil62}

Teorie jako taková je založena na propracovaném systému
onomaziologických a~pojmových kategoriích, které obsahy vědomí
strukturují, tedy zobecňují nebo konkretizují.
\parencite{enc-onomaz-kateg17} Tato problematika je dále rozebírána
v~podkapitole \ref{onomaziologickuxe9-kategorie-a-jejich-klasifikace}.

\hypertarget{zpux16fsoby-a-prostux159edky-tvoux159enuxed-novuxfdch-pojmenovuxe1nuxed}{%
\section{Způsoby a~prostředky tvoření nových
pojmenování}\label{zpux16fsoby-a-prostux159edky-tvoux159enuxed-novuxfdch-pojmenovuxe1nuxed}}

Potřebu nových pojmenování lze naplnit různými způsoby, prvním z~nich je
využití již vytvořených pojmenování z~cizích jazyků -- k~tomuto
přejímání může docházet z~vícero důvodů, například když nemá domácí
jazyk pro daný pojem z~nějaké příčiny výraz (\emph{whiskey}), ale také
může jít o~nějakou formu jazykové módy (\emph{cool}) nebo eufemismu
(\emph{toaleta}).~\parencite[19]{dokulil62}

Druhým způsobem, jak vytvářet nová pojmenování, je tvorba nových výrazů
z~vlastní slovní zásoby daného jazyka. Podle Dokulila lze v~této oblasti
mluvit o~tvoření pojmenování víceslovných a~jednoslovných.
U~víceslovných novotvarů jde typicky jen o~novou kombinaci slov již
existujících (\emph{strojový překlad}), tento typ tvorby nových
pojmenování je v~českém jazyce nejčastější a~nejjednodušší, nicméně
existují důkazy, že čeština dává z~důvodu své flektivní povahy přednost
právě druhému typu. Ten je založen na tom, že nová jednoslovná
pojmenování vznikají prostřednictvím morfologických změn ze slov už
vytvořených -- může se tak jednat buď o~skládání slov (kompozici) nebo
odvozování slov (derivaci).~\parencite[21]{dokulil62}

\hypertarget{kompozice}{%
\subsection{Kompozice}\label{kompozice}}

Charakteristickým rysem výrazů vzniklých skládáním slov je to, že
obsahují více než jeden slovní základ (\emph{čern-o-vlasý}), někdy tedy
bývá kompozice označována za přechodný způsob mezi tvořením slov
víceslovných a~jednoslovných. Za specifický typ kompozit bývají
označovány spřežky (nevlastní složeniny) -- tato slova vznikla
spřáhnutím slov, která se často objevovala v~nějak slovním spojení
(\emph{boj-e-schopný}). Jejich vlastností je, že je lze za určitých
podmínek zpětně rozpojit do separátního stavu (\emph{schopný boje}).
\parencite[22]{dokulil62}

Ostatní složená slova jsou nerozložitelná do víceslovného pojmenování
a~jejich první člen nebývá hodnocen jako úplný tvar slova, taktéž bývá
mezi tvary přítomen kompoziční vokál \emph{o}, \emph{e} nebo \emph{i}.
Skládání slov není ve slovanských jazycích častým jevem, a~proto
i~v~češtině za nejdůležitější postup ve slovotvorbě považujeme derivaci.
\parencite[22]{dokulil62}

\hypertarget{derivace}{%
\subsection{Derivace}\label{derivace}}

Odvozování slov je založeno na tvoření slov od slov jiných (označujeme
je jako základové) prostřednictvím změny v~morfologické stavbě -- tyto
změny bývají způsobeny určitými odvozovacími prostředky (formanty).
\parencite[93]{dokulil00}

Podle pozice jednotlivých formantů můžeme vydělit několik základních
slovotvorných postupů:

\begin{itemize}
\tightlist
\item
  prefixace -- před slovo základové je umístěn slovotvorný morfém
  (prefix), jenž u~většiny slovních druhů nemění mluvnické
  charakteristiky (\emph{vinný} --\textgreater{} \emph{ne-vinný}), pouze
  u~verb může a~nemusí prefixace měnit slovesný vid (imperfektivum
  \emph{dělat} --\textgreater{} perfektivum \emph{dodělat}, ale
  imperfektivum \emph{dělávat} --\textgreater{} imperfektivum
  \emph{dodělávat});
\item
  sufixace -- za slovo základové je umístěn slovotvorný morfém (sufix),
  ten se vždy připojuje za základ slova, nikoliv za celé základové slovo
  (\emph{prav-ic-e}). Sufixace je nejdůležitější slovotvorný postup
v~češtině~\parencite[23]{dokulil62} a~je spjata se souborem koncovek
  určitého paradigmatu~\parencite[93]{dokulil00};
\item
  reflexivizace -- odvozování zvratných sloves pomocí volných zvratných
  formantů \emph{se}, \emph{si} (\emph{bavit se});
\item
  postfixace -- odvozování od úplného slovního tvaru pomocí zvláštní
  přípony (\emph{koho-si});
\item
  deprefixace -- odsunutí prefixu (\emph{poslat} --\textgreater{}
  \emph{slát});
\item
  desufiaxce -- odsunutí sufixu (\emph{plamen} --\textgreater{}
  \emph{plam})\footnote{V rámci aglutinačních jazyků (korejština, finština atd.) by se dalo taktéž mluvit o~infixaci, kde se za infix považuje takový afix, jenž se vyskytuje uvnitř kořenu slova, nicméně v~českém jazyce se tento jev nevyskytuje.~\parencite{enc-morfem17}}.
 ~\parencite[93--94]{dokulil00}
\end{itemize}

Na závěr této podkapitoly je nutné poznamenat, že se výše vypsané
slovotvorné způsoby mohou různě kombinovat~\parencite[93]{dokulil00},
například podstatné jméno \emph{výsadek} je odvozeno od slova \emph{sad}
nebo \emph{sázet}, a~to jak pomocí prefixu \emph{vý-}, tak
prostřednictvím sufixu \emph{-ek}.

Ve zbytku práce budeme teorii rozebírat výhradně na příkladech derivace
-- jednak z~příčiny zmíněné výše, tedy že je v~českém jazyce tento
způsob nejčastější, a~jednak z~ryze pragmatického důvodu, protože
v~praktické části popisujeme aplikaci, která pracuje výhradně
s~derivačními rysy.

\hypertarget{slovotvornuxe9-vztahy}{%
\section{Slovotvorné vztahy}\label{slovotvornuxe9-vztahy}}

Pro popis slovotvorných vztahů přišla OTS s~dvěma základními termíny,
jde o~vztah fundace a~motivace.

\hypertarget{fundace}{%
\subsection{Fundace}\label{fundace}}

Vztah fundace (zakládání se jednoho slova na druhém) je založen na tom
principu, že slovo, které je významově i~formálně složitější (slovo
fundované), se zakládá na slově jiném, které je pro něj základové (slovo
fundující), a~tím pádem zpravidla významově i~formálně jednodušší.
\parencite[95]{dokulil00}

Tento vztah je základním slovotvorným vztahem, a~tedy pokud u~nějaké
dvojice slov chybí, nelze pak tvrdit, že tato slova spolu jakkoliv
slovotvorně souvisí. Příkladem fundace může být slovo \emph{učitel},
které je fundované (je od něj odvozeno) fundujícím slovem \emph{učit}.

Na dvojicích slov ve vztahu fundace se mohou dále tvořit složitější
a~rozvětvenější vztahy, jde o:

\begin{itemize}
\tightlist
\item
  slovotvorné svazky -- jedno slovo má vícečlennou množinu fundovaných
  anebo fundujících slov (Slovo \emph{list} má více fundovaných slov --
  \emph{lístek}, \emph{listopad}, \emph{listovat} atd. Slovo
  \emph{listopad} má větší počet slov fundujících -- \emph{list},
  \emph{padat}, \emph{pád} atd.);
\item
  slovotvorné řády -- vztah fundace platí i~mezi sousedními členy
  (řetězec slov \emph{učit} --\textgreater{} \emph{učitel}
  --\textgreater{} \emph{učitelka});
\item
  slovotvorná hnízda (čeledě) -- komplexnější síť vztahů, kterou vytváří
  jedno slovo základové prostřednictvím většího množství slovotvorných
  řad a~svazků.~\parencite[12--13]{dokulil62}
\end{itemize}

\hypertarget{motivace}{%
\subsection{Motivace}\label{motivace}}

Vztah motivace mezi dvěma slovy je pak především o~určité významové
odvozenosti, tedy že základové slovo (motivující) sémanticky předurčuje
slovo motivované, jehož význam lze takto ozřejmit.
\parencite[96]{dokulil00} Příklad motivace může být výraz
\emph{mladice}, který je motivován substantivem \emph{mladík} (ve
významu \emph{mladý muž}) a~adjektivem \emph{mladý}.
\parencite[110]{dokulil62}

Obecně označujeme slova motivovaná za popisná (\emph{učitel}), protože
je lze popsat významem slova motivujícího (\emph{učit}). Výrazy, které
tuto vlastnost nemají pak definujeme jako značková.
\parencite[96]{dokulil00}

Vztahy fundace a~motivace mají typicky jasný směr od slova základového
k~odvozenému, nicméně ne vždy může být směr fundace/motivace zřejmý, proto
v~určitých případech považujeme vztah za oboustranný. Obecně jde
o~taková slova, kterým chybí formální příznak odvozenosti, tedy jasně
pozorovatelná morfologická změna, v~češtině tak nejčastěji derivační
sufix.~\parencite[96]{dokulil00}

Pokud tento tento formální příznak odvozenosti zcela chybí, lze směr
fundace/motivace často odhadnout na základě analogie, tedy například
u~dějového jména \emph{zloba} ve vztahu se slovesem \emph{zlobit} nelze
vyčíst žádný formální prostředek, který by směr odvozenosti určoval. Lze
ale využít analogie, tedy odvozenosti jiných dějových jmen, u~kterých
tento prostředek existuje (\emph{zlobit} --\textgreater{}
\emph{zlobení}, \emph{zabíjet} --\textgreater{} \emph{zabíjení}), a~tak
ozřejmit směr fundace/motivace, v~tomto případě tedy \emph{zlobit}
--\textgreater{} \emph{zloba}.~\parencite[96]{dokulil00}

V~okamžiku, kdy nelze využít analogie, pracujeme s~oboustrannou
fundací/motivací -- jde o~takové dvojice slov, kde:

\begin{itemize}
\tightlist
\item
  má každý výraz svůj vlastní slovotvorný formant (\emph{jablko}
  \textless{}--\textgreater{} \emph{jabloň});
\item
  tento formant u~obou slov chybí (\emph{doma}
  \textless{}--\textgreater{} \emph{domů});
\item
  je formální a~sémantické kritérium v~rozporu (\emph{něha}
  \textless{}--\textgreater{} \emph{něžný}).~\parencite[96]{dokulil00}
\end{itemize}

\hypertarget{slovotvornuxfd-a-lexikuxe1lnuxed-vuxfdznam}{%
\subsection{Slovotvorný a~lexikální
význam}\label{slovotvornuxfd-a-lexikuxe1lnuxed-vuxfdznam}}

V~rámci problematiky slovotvorných vztahů si je taktéž zapotřebí vydělit
dva základní typy významu, se kterými teorie pracuje. Prvním z~nich je
význam slovotvorný (strukturní), který je založen na významech
jednotlivých slovotvorných složek, to znamená, že jej lze do určité míry
predikovat podle morfému/ů, o~který/é je slovo odvozené obohaceno anebo
zkráceno (viz \ref{derivace}) oproti slovu základovém.
\parencite{enc-slovot-vyznam17}

Lexikální význam je na rozdíl od významu slovotvorného záležitostí
konvence a~úzu, tedy se může od strukturního významu vzdálit
(lexikalizovat), a~tím získat význam nový, jenž nelze vyčíst ze
slovotvorné struktury daného slova a~nelze ho tak nijak predikovat.
\parencite{enc-slovot-vyznam17} Jak bylo již výše naznačeno, oba typy
významu se mohou více či méně rozejít, tuto divergenci pěkně demonstruje
výraz \emph{krejčí}, u~něhož je strukturní význam „ten, kdo krájí``
a~lexikální „ten, kdo šije šaty``. Nicméně tak existují i~případy, kdy se
oba typy významu rovnají (například u~slova \emph{učitel}) anebo kdy je
strukturní význam oproti lexikálnímu obecnější. (Např. výraz
\emph{vnímatel}, jehož definice je v~SSJČ „ten, kdo uvědoměle vnímá
umělecké dílo``~\parencite{ssjc}, zde je tedy lexikální význam užší než
strukturní, který zní následovně: „ten, kdo vnímá``.) Poslední případ,
jenž není v~českém jazyce hojně zastoupen, je situace, kdy je lexikální
význam obecnější (například XXX).

Hlavní rozdíl mezi těmito dvěma významu je tedy ten, že slova značková
mají pouze lexikální význam, jelikož jsou typicky na počátku případného
derivačního řetězce, a~nemohou proto nabývat významu slovotvorného.
Například značkové slovo \emph{les} má pouze význam lexikální, jenž lze
najít ve výkladových slovnících, na druhou stranu výraz \emph{lesnatý}
je již slovem odvozeným, a~proto u~něj můžeme vydělit oba druhy významu.
Lexikální význam zase z~výkladového slovníku, ale nově i~slovotvorný,
který je dán určitým afixem, v~tomto případě sufixem \emph{-natý}, který
je ve slovníku afixu definován takto: „mající v~hojné míře to, co
označuje základové substantivum``~\parencite{simandl2016}.

\hypertarget{morfologickuxe1-stavba-slov}{%
\section{Morfologická stavba
slov}\label{morfologickuxe1-stavba-slov}}

OTS nepojímá morfologickou stavbu slova jako posloupnost jednotlivých
morfémů, které mají určitou funkci v~celkovém významu slova, ale jako
stavbu tvarotvornou a~slovotvornou -- tím je myšleno, že se na
morfologickou strukturu slova dívá ze dvou různých perspektiv.
\parencite[118]{dokulil62}

První z~nich, stavba tvarotvorná, je založena na rozdělení slovního
tvaru na dvě složky, tedy na tvarotvorný základ (lexikální složka,
kořen) a~tvarotvorný formant (gramatická složka). Například u~slova
\emph{bažina} můžeme vydělit tvarotvorný základ \emph{bažin-} a~formant
\emph{-a}, nicméně tvarotvorný formant nemusí být pouze jednočlenný,
protože v~sobě obsahuje (pokud se v~daném slově vyskytuje) kmenotvornou
příponu\footnote{Terminologicky se pak tvarotvorný základ spolu s~kmenotvornou příponou označuje za kmen slova.}.
Mějme druhý příklad, slovo \emph{kuřata}, kde za tvaroslovný základ
považujeme morfém \emph{kuř-}, za kmenotvornou příponu \emph{-at-} a~za
celý slovotvorný formant vícečlennou složku \emph{-ata} -- tím se tento
formant odlišuje od tradičního pojetí koncovky.
\parencite{enc-tvar-zaklad17}

Slovotvorná stavba slova se pak analogicky dělí na dvě složky, ale u~ní
bereme zřetel právě na slovotvorné charakteristiky. Za slovotvorný
základ (bázi) slova typicky považujeme buď celé slovo základové (případy
prefixace nebo reflexivizace -- \emph{ne-\textbf{pěkný}},
\emph{\textbf{bavit} se}) nebo jen část fundujícího slova (nejčastěji
při sufixaci -- \emph{\textbf{prav}-i-ce}), důležité zde je, že tato
část reprezentuje význam základového slova.
\parencite{enc-slov-zaklad17}

K~tomuto základu se pak připojuje slovotvorný formant, jenž obsahuje
vše, čím se formálně liší slovo fundované od slova základového. Častým
jevem v~rámci slovotvorné stavby bývají alternace hlásek, jinými slovy
situace, kdy se pravidelně střídají fonémy v~témže morfému podle daného
kontextu.~\parencite[97]{dokulil00}

Na příkladě slova \emph{učitelka} si demonstrujeme oba přístupy
k~morfologické stavbě slova. Výraz \emph{učitelka} má slovotvorný základ
\emph{učitel} (slovo fundující) a~slovotvorný formant \emph{-k} (morfém,
který mění slovotvorný význam). Tvarotvorným základem tohoto slova pak
je tvar \emph{učitelk-} a~tvarotvorným formantem je pádová koncovka
\emph{-a} (na základě které je dané substantivum zasazené do určitého
deklinačního paradigmatu).

\hypertarget{onomaziologickuxe9-kategorie-a-jejich-klasifikace}{%
\section{Onomaziologické kategorie a~jejich
klasifikace}\label{onomaziologickuxe9-kategorie-a-jejich-klasifikace}}

Jak bylo v~předešlých podkapitolách poznamenáno, OTS je založena na
teorii pojmenování, tedy pracuje s~předpokladem, že se jednotlivé pojmy
před samotnou realizací musí nejprve v~rámci daných jazykových
zákonitostí určitého jazyka utřídit, zobecnit, nebo naopak
konkretizovat. Tyto onomaziologické kategorie se pak vzájemně odlišují
podle jednotlivých pojmenovávacích vlastností a~přístupů.
\parencite[29]{dokulil62}

\hypertarget{pojmovuxe9-kategorie}{%
\subsection{Pojmové kategorie}\label{pojmovuxe9-kategorie}}

Teorie dělí jednotlivé pojmy do čtyř základních pojmových (obsahových)
kategorií -- jde o~kategorie substance (podstata), vlastnosti (kvalita),
děje (proces) a~okolnosti (způsob či míra příznaku), které odpovídají
základním slovním druhům, jimiž jsou substantiva, adjektiva, verba
a~adverbia, a~proto tyto kategorie můžeme nazývat i~jako slovnědruhové.
\parencite[102]{dokulil00}

\hypertarget{pojmenovuxe1vacuxed-kategorie}{%
\subsection{Pojmenovávací
kategorie}\label{pojmenovuxe1vacuxed-kategorie}}

Onomaziologické kategorie lze pak popsat jako „strukturu pojmových
kategorií``, protože v~rámci nich jsou vždy pojmové kategorie v~nějakém
vztahu -- jde tedy o~vztah mezi pojmovou bází a~příznakem.
\parencite{enc-onomaz-kateg17}

Pod pojmovou bází si lze představit výsledek počátečního zařazení jevu,
který má být pojmenován, do jedné ze čtyř z~pojmových kategorií.
Výsledné pojmenování je pak určeno určitým onomaziologickým příznakem --
příkladem může být výraz \emph{bažinatý}, jehož pojmovou bázi je
vlastnost (takové místo, kde jsou bažiny) a~které je určeno příznakem
substance (\emph{bažina}), výsledné pojmenování je pak utvořeno
vzájemným vztahem těchto dvou složek.~\parencite[29]{dokulil62}

Vztahy mezi pojmovou bází a~příznakem při utváření nových slov mají více
druhů, proto si níže ty nejdůležitější více rozvedeme. Nebudeme u~nich
uvádět příklady všech kombinací napříč všemi pojmovými třídami, ale
pomocí vybraných výrazu budeme demonstrovat jejich podstatu.

\hypertarget{mutace}{%
\subsubsection{Mutace}\label{mutace}}

Významnou pojmenovávací kategorií je kategorie mutační (relační),
u~nových pojmenování vzniklých mutací dochází k~výrazné proměně obsahu
výchozího pojmu~\parencite[102]{dokulil00}, tuhle změnu určuje
onomaziologický příznak, jenž je také určujícím rysem, který výsledný
pojem od ostatních prvků dané třídy odlišuje.~\parencite{enc-mutace17}

Za příklad si můžeme uvést pojem výrazu \emph{učitel} (ten, kdo učí),
jde o~kategorii substance, která je vymezena vztahem k~ději (konkrétně
k~ději \emph{učit}). Tento dějový příznak tedy plně odlišuje výsledný
pojem v~rámci třídy podobných jevů, v~tomto konkrétním případě jde
o~třídu činitelských jmen (viz XXX).

\hypertarget{modifikace}{%
\subsubsection{Modifikace}\label{modifikace}}

Na rozdíl od mutace je kategorie modifikační (obměnná) založena na
přidání určitého dodatečného příznaku, to znamená, že výrazně nemění
obsah výchozího pojmu, ale jen jej modifikuje (nedochází tedy ke změněně
pojmové kategorie).~\parencite[102]{dokulil00}

Například u~pojmenování založených na kategorii substance je tento
příznak typicky příznakem deminuce (\emph{hotel} --\textgreater{}
\emph{hotýlek}), augmentace (chlap --\textgreater{} chlapák) nebo
přirozeného rodu (\emph{pes} --\textgreater{} \emph{psice}).

\hypertarget{transpozice}{%
\subsubsection{Transpozice}\label{transpozice}}

Poslední hlavní onomaziologickou kategorií je kategorie modifikační
(převodová), u~níž je vždy obsah výchozí pojmové kategorie přesunut do
pojmové kategorie jiné. Jde o~případy, kdy výsledná pojmová kategorie
není určena pojmenovávacím příznakem, protože tím je vlastně výsledný
pojem sám o~sobě -- z~toho důvodu dochází pouze ke změně pojmové třídy.
\parencite[103]{dokulil00}

Tuto kategorii můžeme sledovat např. u~zpředmětnění vlastností
(\emph{marnivý} --\textgreater{} \emph{marnivost}) nebo dějů
(\emph{jezdit} --\textgreater{} \emph{jízda}), zvlastnostnění dějů
(\emph{myslet} --\textgreater{} \emph{myslící}) nebo zdějovění
vlastností (\emph{modrý} --\textgreater{} \emph{modrat se}).

\hypertarget{slovotvornuxe9-tux159uxeddy-a-typy}{%
\section{Slovotvorné třídy
a~typy}\label{slovotvornuxe9-tux159uxeddy-a-typy}}

V~oblasti klasifikace slovotvorného systému pracuje OTS s~dvěma
základními termíny -- slovotvorný typ a~slovotvorná třída.

Slovotvorná třída je považována za základní jednotku slovotvorného
systému a~hierarchicky je bezprostředně nadřazená slovotvorným typům.
Tyto jednotky vydělujeme podle několika charakteristik, jednak podle
slovnědruhové třídy (\ref{pojmovuxe9-kategorie}), dále podle
slovotvorného základu (\ref{morfologickuxe1-stavba-slov}) a~nakonec dle
slovotvorného významu
(\ref{slovotvornuxfd-a-lexikuxe1lnuxed-vuxfdznam}).
\parencite[107]{dokulil00}

Tyto třídy jsou pak různé jak pro jednotlivé pojmenovací kategorie, tak
pro každou slovnědruhou třídu. Tedy v~rámci transpozice u~substantiv tak
můžeme vyčlenit za slovotvornou třídu například \emph{názvy vlastností},
\emph{názvy dějů} nebo \emph{nositele vlastností}, u~téhož slovního
druhu u~modifikace \emph{jména přechýlená} či \emph{jména
hromadná/jednotlivin} a~u~mutace např. \emph{jména podle příslušníků}
nebo \emph{jména podle původu a~látky} atd. Stejně bychom dále mohli
postupovat i~u~adjektiv, verb a~adverbií.

Slovotvorné typy pak vždy spadají do určité slovotvorné třídy a~vydělují
se podle slovotvorných formantů, například ve slovotvorné třídě
\emph{jmen činitelských} máme velké množství jednotlivých slovotvorných
typů -- jde např. o~slovotvorné typy se slovotvorným formantem (dále jen
typ) \emph{-tel}, \emph{-č} nebo \emph{-ák}.~\parencite[108]{dokulil00}
Jednotlivých formantů je napříč různými třídami velké množství, a~proto
budeme v~kapitole XXX detailněji rozebírat pouze a~jen ty, které jsou
pro praktickou část této práce relevantní.
