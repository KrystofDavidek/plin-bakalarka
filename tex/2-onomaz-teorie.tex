\hypertarget{onomaziologickuxe1-teorie-slovotvorby}{%
\chapter{Onomaziologická teorie
slovotvorby}\label{onomaziologickuxe1-teorie-slovotvorby}}

\hypertarget{zuxe1kladnuxed-charakteristika}{%
\section{Základní
charakteristika}\label{zuxe1kladnuxed-charakteristika}}

Jak bylo v~minulé kapitole naznačeno, onomaziologická teorie slovotvorby
(dále jen OTS) Miloše Dokulila výrazně ovlivnila přístup ke slovotvorbě
jako takové, protože si kladla za cíl popsat proces tvoření slov soudobé
češtiny a~zároveň definovat hranici s~dalšími lingvistickými
disciplínami jako jsou morfologie či lexikologie -- jedná se tedy o~ryze
synchronní přístup, který je založen na onomaziologii (obecné teorii
pojmenovávání).~\parencite{enc-ots17}

Onomaziologie zkoumá takové motivace, postupy a~prostředky,
prostřednictvím kterých jsou v~daném jazyce vyjadřovány určité obsahy.
Na rozdíl od sémaziologie postupuje od obsahu k~formě, to znamená, že
vychází od pojmu k~pojmenování, kterým je daný pojem vyjádřen.
\parencite{enc-onomaz17} Díky těmto termínům OTS terminologicky vyřešila
problematiku slov \emph{pojem} a~\emph{slovo}, která byla dříve často
nesystematicky zaměňována.~\parencite[267]{rousinova07}

Po metodické stránce Dokulil u~tvoření slov rozlišuje aspekt genetický,
který je zaměřený na vlastní slovotvorné procesy, a~na aspekt funkčně
strukturní, jenž sleduje výsledek těchto procesů (\emph{utvářenost
slov}). Nicméně jak sám autor teorie potvrzuje, je mezi těmito přístupy
těsná spojitost.~\parencite[9]{dokulil62}

Teorie jako taková je založena na propracovaném systému
onomaziologických a~pojmových kategoriích, které obsahy vědomí
strukturují, tedy zobecňují nebo konkretizují. Tato problematika je dále
rozebírána v~XXX.~\parencite{enc-onomaz-kateg17}

\hypertarget{zpux16fsoby-a-prostux159edky-tvoux159enuxed-novuxfdch-pojmenovuxe1nuxed}{%
\section{Způsoby a~prostředky tvoření nových
pojmenování}\label{zpux16fsoby-a-prostux159edky-tvoux159enuxed-novuxfdch-pojmenovuxe1nuxed}}

Potřebu nových pojmenování lze naplnit různými způsoby, prvním z~nich je
využití již vytvořených pojmenování z~cizích jazyků -- k~tomuto
přejímání může docházet z~vícero důvodů, například když nemá domácí
jazyk pro daný pojem z~nějaké příčiny výraz (\emph{whiskey}), ale také
může jít o~nějakou formu jazykové módy (\emph{cool}) nebo eufemismu
(\emph{toaleta}).~\parencite[19]{dokulil62}

Druhým způsobem, jak vytvářet nová pojmenování, je tvorba nových výrazů
z~vlastní slovní zásoby daného jazyka. Podle Dokulila lze v~této oblasti
mluvit o~tvoření pojmenování víceslovných a~jednoslovných.
U~víceslovných novotvarů jde typicky jen o~novou kombinaci slov již
existujících (\emph{strojový překlad}), tento typ tvorby nových
pojmenování je v~českém jazyce nejčastější a~nejjednodušší, nicméně
existují důkazy, že čeština dává z~důvodu své flektivní povahy přednost
právě druhému typu. Ten je založen na tom, že nová jednoslovná
pojmenování vznikají prostřednictvím morfologických změn ze slov už
vytvořených -- může se tak jednat buď o~skládání slov (kompozici) nebo
odvozování slov (derivaci).~\parencite[21]{dokulil62}

\hypertarget{kompozice}{%
\subsection{Kompozice}\label{kompozice}}

Charakteristickým rysem výrazů vzniklých skládáním slov je to, že
obsahují více než jeden slovní základ (\emph{čern-o-vlasý}), někdy tedy
bývá kompozice označována za přechodný způsob mezi tvořením slov
víceslovný a~jednoslovných. Za specifický typ kompozit bývají označovány
spřežky (nevlastní složeniny) -- tato slova vznikla spřáhnutím slov,
která se často objevovala v~nějak slovním spojení
(\emph{boj-e-schopný}). Jejich vlastností je, že je lze za určitých
podmínek zpětně rozpojit do separátního stavu (\emph{schopný boje}).
\parencite[22]{dokulil62}

Ostatní složená slova jsou nerozložitelná do víceslovného pojmenování
a~jejich první člen nebývá hodnocen jako úplný tvar slova, taktéž bývá
mezi tvary přítomen kompoziční vokál \emph{o}, \emph{e} nebo \emph{i}.
Skládání slov není ve slovanských jazycích častým jevem, a~proto
i~v~češtině za nejdůležitější postup ve slovotvorbě považujeme derivaci.
\parencite[22]{dokulil62}

\hypertarget{derivace}{%
\subsection{Derivace}\label{derivace}}

Odvozování slov je založeno na tvoření slov od slov jiných (označujeme
je jako základové) prostřednictvím změny v~morfologické stavbě -- tyto
změny bývají způsobeny určitými odvozovacími prostředky (formanty).
\parencite[93]{dokulil00}

Podle pozice jednotlivých formantů můžeme vydělit několik základních
slovotvorných postupů:

\begin{itemize}
\tightlist
\item
  prefixace -- před slovo základové je umístěn slovotvorný morfém
  (prefix), jenž kromě slovesného vidu nemění mluvnické charakteristiky
  (\emph{ne-vinný}),
\item
  sufixace -- za slovo základové je umístěn slovotvorný morfém (sufix),
  ten se vždy připojuje za základ slova, nikoliv za celé základové slovo
  (\emph{prav-ic-e}). Sufixace je nejdůležitější slovotvorný postup
v~češtině~\parencite[23]{dokulil62} a~je spjata se souborem koncovek
  určitého paradigmatu~\parencite[93]{dokulil00},
\item
  reflexivizace -- odvozování zvratných sloves pomocí volných zvratných
  formantů \emph{se}, \emph{si} (\emph{bavit se}),
\item
  postfixace -- odvozování od úplného slovního tvaru pomocí zvláštní
  přípony (\emph{koho-si}),
\item
  deprefixace -- odsunutí prefixu (\emph{poslat} --\textgreater{}
  \emph{slát}),
\item
  desufiaxce -- odsunutí sufixu (\emph{plamen} --\textgreater{}
  \emph{plan}).~\parencite[93--94]{dokulil00}
\end{itemize}

Na závěr této podkapitoly je nutné poznamenat, že se výše vypsané
slovotvorné způsoby mohou různě kombinovat~\parencite[93]{dokulil00},
například podstatné jméno \emph{výsadek} je odvozeno od slova \emph{sad}
nebo \emph{sázet} jak pomocí prefixu \emph{vý}, tak prostřednictvím
sufixu -ek.

\hypertarget{lexikuxe1lnuxed-a-strukturuxe1lnuxed-vuxfdznam}{%
\section{Lexikální a~strukturální
význam}\label{lexikuxe1lnuxed-a-strukturuxe1lnuxed-vuxfdznam}}

\hypertarget{slovotvornuxe9-vztahy}{%
\section{Slovotvorné vztahy}\label{slovotvornuxe9-vztahy}}

\hypertarget{fundace}{%
\subsection{Fundace}\label{fundace}}

\hypertarget{motivace}{%
\subsection{Motivace}\label{motivace}}

\hypertarget{onomaziologickuxe9-kategorie-a-jejich-klasifikace}{%
\section{Onomaziologické kategorie a~jejich
klasifikace}\label{onomaziologickuxe9-kategorie-a-jejich-klasifikace}}

\hypertarget{substance-vlastnost-dux11bj-okolnost}{%
\subsection{Substance, vlastnost, děj,
okolnost}\label{substance-vlastnost-dux11bj-okolnost}}

\hypertarget{mutace}{%
\subsection{Mutace}\label{mutace}}

\hypertarget{modifikace}{%
\subsection{Modifikace}\label{modifikace}}

\hypertarget{transpozice}{%
\subsection{Transpozice}\label{transpozice}}

\hypertarget{slovotvornuxe9-tux159uxeddy-a-typy}{%
\section{Slovotvorné třídy
a~typy}\label{slovotvornuxe9-tux159uxeddy-a-typy}}

\hypertarget{definice}{%
\subsection{Definice}\label{definice}}

\hypertarget{dux11blenuxed}{%
\subsection{Dělení}\label{dux11blenuxed}}
