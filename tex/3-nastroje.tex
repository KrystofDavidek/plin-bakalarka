\hypertarget{nuxe1stroje-pro-pruxe1ci-s-derivacuxed}{%
\chapter{Nástroje pro práci
s~derivací}\label{nuxe1stroje-pro-pruxe1ci-s-derivacuxed}}

V~rámci české slovotvorby vzniklo nemalé množství počítačových programů,
prostřednictvím kterých lze dosahovat různých výzkumných, edukačních či
komerčních výsledků. Proto si v~první části této kapitoly popíšeme
nejvýznamnější softwarové nástroje, které se využívají pro práci
s~derivací v~českém prostředí. V~druhé části si pak hlouběji představíme
derivační síť DeriNet, na níž je postaveno řešení praktické části této
bakalářské práce.

\hypertarget{pux159ehled-nuxe1strojux16f}{%
\section{Přehled nástrojů}\label{pux159ehled-nuxe1strojux16f}}

\hypertarget{morfio}{%
\subsection{Morfio}\label{morfio}}

Webová aplikace Morfio je jedním z~projektů Českého národního korpusu
(veřejně přístupný bez nutnosti registrace), která „slouží k~odhadování
rozsahu a~produktivity slovotvorných modelů v~češtině na základě
korpusových dat``~\parencite{cvrcek13}. Jde tedy o~systém, který se
snaží ve zvoleném korpusu (SYN2000, SYN2010 nebo SYN2015) najít takové
n-tice formálně podobných slov, které se shodují určitým slovotvorným
základem a~liší se specifickým slovotvorným formantem (těch může být
i~více, navíc je zde reflektována problematika hláskových alternací).
Nástroj je tedy vhodný spíše jako výzkumná pomůcka nežli jako prostředek
k~nacházení slovotvorných vztahů mezi slovy, protože při manipulaci
s~korpusovými daty, jež nejsou nijak sémanticky označkována, může docházet
k~chybám například z~důvodu homonymie.~\parencite{cvrcek13}

\hypertarget{ajka}{%
\subsection{Ajka}\label{ajka}}

Dalším nástrojem je morfologický analyzátor Ajka (taktéž veřejně
přístupný bez nutnosti registrace), jehož hlavní složkou je analýza
flektivní morfologie. To znamená, že obsahuje rozsáhlý systém vzorů
spolu se sadami určitých koncovek a~morfologických značek. Ve webovém
rozhraní je možnost vstupní text buď segmentovat na jednotlivé
morfologické segmenty, analyzovat z~hlediska morfologických
charakteristik (vstupnímu slovu je přiřazena určitá morfologická
značka), nebo vyhledat existující akcentovaný výraz (například pro vstup
\emph{blázen} je výstupem výraz \emph{blažen}). Nástroj nicméně
akcentuje i~složku derivační, a~to ve formě hierarchického systému
morfologických paradigmat, který slouží pro zachycení různých úrovní
derivační morfologie.~\parencite{ajka}

V~průběhu času vznikl z~potřeby efektivnějšího zpracování textu
z~morfologického analyzátoru Ajka nástroj Majka, který používá stejná
jazyková data, ale kompletně proměnil jejich formát. Analyzátor Majka
taktéž využívá efektivnější algoritmus (vyvinutý Janem Daciukem), díky
kterému bylo docíleno k~šestinásobnému zrychlení zpracování.
\parencite{majka}

\hypertarget{deriv}{%
\subsection{Deriv}\label{deriv}}

Třetím softwarovým řešením zabývající se slovní derivací je nástroj
Deriv (přístupný pouze po přihlášení), jehož primárním cílem je testovat
možnosti automatické slovotvorné analýzy. Jeho webového rozhraní se
skládá ze dvou základních funkcí -- vyhledávání podle formálního zadání
a~kategorizace vyhledaných dat. Samotný Deriv je založený na
automatickém morfologickém analyzátoru Ajka (později Majka), což
znamená, že využívá jeho morfologický slovník kmenů, v~němž vyhledává
prostřednictvím morfologických značek a~regulárních
výrazů\footnote{Nástroj využívá regulární výrazy programovacího jazyka Perl verze 5.10 a~novější.}.
Výsledky vyhledávání jsou na rozdíl od ostatních nástrojů propojeny
s~českými výkladovými slovníky (SSČ, SSJČ, PSJČ) a~s~některými českými
korpusy (konkrétně CzTenTen, SYN2000).~\parencite{deriv}

\hypertarget{derivancze}{%
\subsection{Derivancze}\label{derivancze}}

Předposledním rozebíraným nástrojem je derivační analyzátor Derivancze
(volně přístupný bez předchozí registrace), který analyzuje slovotvorné
vztahy mezi slovy. Svojí funkcionalitou je podobný derivační síti
DeriNet, ale na rozdíl od ní zohledňuje sémantický aspekt nad formálním.
To znamená, že pokud se formální a~významový aspekt derivačního vztahu
liší, tak je vyžadováno explicitní označkovaní, díky kterému pak dochází
ke konzistenci napříč daty. Nástroj využívá sedmnácti značek označující
typ sémantického vztahu -- může jít například o~značku \emph{k1ag},
která označuje vztah odvození verba směrem k~činitelskému jménu. Kvůli
tomuto přístupu je Derivancze problematické použít v~rámci větších
sofistikovanějších aplikací (například automatické generování textu),
protože by bylo zapotřebí dodat více informací o~typech jednotlivých
slovotvorných vztahů.~\parencite{derivancze}

\hypertarget{derivaux10dnuxed-suxedux165-derinet}{%
\section{Derivační síť
DeriNet}\label{derivaux10dnuxed-suxedux165-derinet}}

Řešení praktické části této bakalářské je postaveno na derivační síti
DeriNet, proto si tento nástroj v~následující podkapitole hlouběji
charakterizujeme a~popíšeme si základní způsoby, jak s~ním lze pracovat.

Derivační síť DeriNet si lze představit jako elektronickou databázi
českých autosémantik (tedy primárně substantiv, adjektiv, verb
a~adverbií), která jsou vzájemně propojena takovými odkazy, jež odpovídají
slovotvornému vztahu derivace mezi slovem základovým a~odvozeným. Tento
systém odkazů si lze tak modelovat prostřednictvím orientovaného grafu,
jehož uzly jsou jednotlivá lemmata (základní slovní tvary) a~hrany pak
spolu s~jejich orientací reprezentují určitý odvozovací proces. Jelikož
má v~této derivační síti každé odvozené slovo odkaz pouze na jedno slovo
základové, lze si tak jednotlivé slovotvorná hnízda (čeledě) představit
jako stromový graf, jehož kořenem je ideálně slovo značkové, tedy takový
výraz, který není nijak motivován.~\parencite{derinet-cz}

Před samotným vznikem tohoto projektu bylo řešeno několik lingvistických
skutečností, které ovlivnily výslednou podobu DeriNetu -- primárně šlo
o~problematiku výběru původních lexémů. Zde existovaly dvě varianty, buď
využít již existujících jazykových dat z~korpusů, nebo poloautomaticky
generovat daná lemmata. Výsledným rozhodnutím bylo držet se úzu
a~extrahovat lexémy z~korpusu (konkrétně šlo o~korpus řady SYN a~bylo
vybráno přibližně dvě stě šedesát tisíc lemmat). Druhou otázkou bylo,
jaký typ slovotvorného vztahu by měl být v~lexikální síti reflektován.
V~tomto případě byla vybrána derivace z~důvodu svého dominantního
postavení v~české slovotvorbě, nicméně je do budoucna plánováno popsat
i~slovotvorné vztahy týkající se skládání slov, což by ale kompletně
pozměnilo architekturu sítě, v~níž má každé základové slovo právě jeden
odkaz na určitý derivát.~\parencite{sevcikova14}

Pro tvorbu samotné derivační sítě, resp. jednotlivých derivačních vztahů
bylo zapotřebí identifikovat dvojice slov základových a~odvozených --
toho bylo docíleno poloautomatickou metodou, v~níž šlo tedy nejprve
o~automatizovaný výběr takových dvojic lemmat, v~rámci kterých měla obě
slova dostatečně dlouhý společný počáteční podřetězec a~zároveň se
lišila specifickým koncovým podřetězcem. Ze čtyř set
nejfrekventovanějších pak bylo ručně vyextrahováno osmnáct dvojic, které
odpovídají popisovaným derivačním vztahům v~české slovní zásobě.
U~těchto dvojic byl taktéž určen směr derivace (typicky bylo za základové
slovo považováno to, které mělo kratší základové lemma).
\parencite{derinet-cz}

Dalším způsobem, pomocí kterého šlo určovat derivační vztahy, byla
extrakce informací z~morfologického slovníku MorFlex CZ -- ten obsahuje
sto dvacet milionů trojic tvořených lemmatem, morfologickou značkou
a~slovním tvarem, příkladem může být trojice ‒ \emph{žížnivý\_,n};
\texttt{AAFS6-\/-\/-\/-3N-\/-\/-\/-}; \emph{nejnežížnivější}.
\parencite{morflex}

Jednotlivá lemmata mohou taktéž obsahovat dodatečnou derivační
a~sémantickou informaci, právě tak slovník řeší například problematiku
lexikální homonymie -- homonymní lemmata jsou označena specifickým
číselným identifikátorem (např. \emph{podle-1} ve významu adverbia
a~\emph{podle-2} jako prepopozice). Derivační informace je ve slovníku
reprezentována prostřednictvím technického sufixu, který je uložen
u~daného lemmatu (např. u~základního tvaru \emph{žíznitelný\_\^{}(}4)* je
derivační informace uložena v~závorce a~vyplývá z~ní, že je slovo
\emph{žíznitelný} odvozeno od slova o~čtyři znaky kratšího, tedy od
slova \emph{žíznit}). Díky datům ze slovníku MorfFlex CZ byl počet
lemmat v~DeriNetu přibližně ztrojnásoben a~zároveň doplněn o~velké
množství derivačních vztahů.~\parencite{sevcikova16}

\begin{figure}[ht]   
    \centering
    \includegraphics[width=.9\textwidth]{derinet-1}  
    \caption{Výsledný derivační strom v~prohlížeči DeriNet Viewer~\parencite{derinet}}
    \label{derinet-1}
 \end{figure}

Aktuální verze derivační sítě DeriNet (1.7) obsahuje přibližně jeden
milion lemmat a~je dostupná skrze dvě webová uživatelská rozhraní.
Jedním z~nich je prohlížeč DeriNet
Viewer\footnote{Autor je Milan Straka -- http://ufal.mff.cuni.cz/derinet/derinet-viewer},
jehož základní funkcí je zobrazení derivačního stromu pro zadané lemma
(srov. obr. \ref{derinet-1}), případně lze vyhledané výsledky roztřídit
podle zvolených charakteristik. Druhý nástroj je DeriNet
Search\footnote{Autor je Jonáš Vidra -- http://ufal.mff.cuni.cz/derinet/search.},
který nabízí vyhledávání ve vlastním dotazovacím jazyce, to znamená, že
je uživatel schopen podle svých vlastních kritérii specifikovat omezení
pro tvar vyhledaného derivačního stromu. Například dotaz
\texttt{{[}pos="V"{]}\ ({[}pos="N"\ lemma="tel\$"{]},\ {[}pos="N"\ lemma="ce\$"{]})}
vyhledá takové derivační stromy, ve kterých je od verba přímo odvozeno
jednak substantivum končící sufixem \emph{-tel}, tak substantivum na
\emph{-ce} (viz obr. \ref{derinet-2}).~\parencite{derinet-cz}

\begin{figure}[ht]   
    \centering
    \includegraphics[width=.9\textwidth]{derinet-2}  
    \caption{Výsledek vyhledávacího dotazu ve vyhledávači DeriNet Search~\parencite{derinet}}
    \label{derinet-2}
 \end{figure}

Další možnost, jak pracovat s~databází DeriNet, je prostřednictvím
jednoduchého datového formátu TSV (anglicky \emph{Tab-Separated Values}
-- jde o~textovou reprezentaci tabulkových dat, které jsou od sebe
odděleny tabulátorem), jenž je zpřístupněn k~volnému stažení pod licencí
Creative Commons Attribution-NonCommercial-ShareAlike 3.0 License.
U~tohoto přístupu se již počítá se základními programátorskými
dovednostmi, protože takto strukturovaná data primárně slouží jako vstup
pro určitý software.~\parencite{derinet-cz}

\begin{verbatim}
692744  superuživatel   superuživatel   N   775428
775428  uživatel    uživatel    N   775440
775440  užívat  užívat_:T_^(*3t)    V   775402
775402  užít    užít_:T V   1006682
1006682 žít žít_:T  V
\end{verbatim}

Každý záznam (řádek) v~tomto souboru obsahuje několik atributů, jde
o~vlastní identifikační číslo, lemma, derivační informaci přejatou
z~morfologického slovníku MorFlex CZ, značku slovního druhu a~u~slov
derivovaných identifikační číslo slova základového -- tím je jednoznačně
vyznačen derivační vztah. Tento formát si můžeme ilustrovat na příkladu
derivačního řetězce \emph{žít} $\rightarrow$ \emph{užít}
$\rightarrow$ \ldots{} $\rightarrow$ \emph{superuživatel}. V~této
formě DeriNetu je reprezentován pěti záznamy, které jsou vzájemně
propojeny identifikačními čísly (poslední atribut odkazuje na první),
pouze u~značkového slovesa \emph{žít} žádný další odkaz neexistuje.
\parencite{derinet} A~právě tento formát DeriNetu slouží jako základ
derivačního slovníku v~rámci praktické části této práce.
