\hypertarget{nuxe1stroje-pro-pruxe1ci-s-derivacuxed-v-ux10deskuxe9m-prostux159eduxed}{%
\chapter{Nástroje pro práci s~derivací v~českém
prostředí}\label{nuxe1stroje-pro-pruxe1ci-s-derivacuxed-v-ux10deskuxe9m-prostux159eduxed}}

Na základě onomaziologické teorie Miloše Dokulila (a~jeho následovníků)
stojí nemalé množství počítačových programů, prostřednictvím kterých lze
dosahovat různých výzkumných, edukačních či komerčních výsledků, proto
si v~první část této kapitoly popíšeme nejvýznamnější softwarové
nástroje, které se využívají pro práci s~derivací v~českém prostředí.
V~druhé části si pak hlouběji představíme derivační síť Derinet, na níž je
postaveno řešení praktické části této bakalářské práce.

\hypertarget{pux159ehled-nuxe1strojux16f}{%
\section{Přehled nástrojů}\label{pux159ehled-nuxe1strojux16f}}

\hypertarget{morfio}{%
\subsection{Morfio}\label{morfio}}

Něco~\parencite{cvrcek13}

\hypertarget{ajka}{%
\subsection{Ajka}\label{ajka}}

\hypertarget{deriv}{%
\subsection{Deriv}\label{deriv}}

\hypertarget{derivancze}{%
\subsection{Derivancze}\label{derivancze}}

Prvním nástrojem, který zde uvádíme, je Něco hezkého
\parencite[516]{pala15}

\hypertarget{derivaux10dnuxed-suxedux165-derinet}{%
\section{Derivační síť
Derinet}\label{derivaux10dnuxed-suxedux165-derinet}}
