\hypertarget{nuxe1stroje-pro-pruxe1ci-s-derivacuxed-v-ux10deskuxe9m-prostux159eduxed}{%
\chapter{Nástroje pro práci s~derivací v~českém
prostředí}\label{nuxe1stroje-pro-pruxe1ci-s-derivacuxed-v-ux10deskuxe9m-prostux159eduxed}}

Na základě onomaziologické teorie Miloše Dokulila (a~jeho následovníků)
vzniklo nemalé množství počítačových programů, prostřednictvím kterých
lze dosahovat různých výzkumných, edukačních či komerčních výsledků,
proto si v~první část této kapitoly popíšeme nejvýznamnější softwarové
nástroje, které se využívají pro práci s~derivací v~českém prostředí.
V~druhé části si pak hlouběji představíme derivační síť Derinet, na níž je
postaveno řešení praktické části této bakalářské práce.

\hypertarget{pux159ehled-nuxe1strojux16f}{%
\section{Přehled nástrojů}\label{pux159ehled-nuxe1strojux16f}}

\hypertarget{morfio}{%
\subsection{Morfio}\label{morfio}}

Webová aplikace Morfio je jedním z~projektů Českého národního korpusu,
která „slouží k~odhadování rozsahu a~produktivity slovotvorných modelů
v~češtině na základě korpusových dat``. Jde tedy o~systém, který se snaží
ve zvoleném korpusu najít takové n-tice slov, které se shodují určitým
slovotvorným základem a~liší se specifickým slovotvorným formantem (těch
může být i~více, navíc je zde reflektována problematika hláskových
alternací.) Nástroj je tedy vhodný spíše jako výzkumná pomůcka než-li
jako prostředek ke tvorbě relevantních lingvistických výstupů, protože
při manipulaci s~korpusovými daty, jež nejsou nijak sémanticky
označkována, může docházet k~chybám například z~důvodu homonymie.
\parencite{cvrcek13}

\hypertarget{morfologickuxe9-analyzuxe1tory-ajka}{%
\subsection{Morfologické analyzátory
Ajka}\label{morfologickuxe9-analyzuxe1tory-ajka}}

Dalším nástrojem je morfologický analyzátor Ajka, jehož hlavní složkou
je analýza flektivní morfologie -- to znamená, že obsahuje rozsáhlý
systém vzorů spolu se sadami určitých koncovek a~morfologických značek.
Ve webovém rozhraní je možnost vstupní text buď segmentovat na
jednotlivé morfologické segmenty, analyzovat z~pohledu určitého
paradigmatu nebo vyhledat existující akcentovaný výraz (například pro
vstup \emph{blázen} je výstupem výraz \emph{blažen}). Nástroj nicméně
akcentuje i~složku derivační, a~to ve formě hierarchického systému
morfologických paradigmat, který slouží pro zachycení všech úrovní
derivační morfologie.~\parencite{ajka}

V~průběhu času vznikl z~potřeby efektivnějšího zpracování textu
z~morfologického analyzátoru Ajka nástroj Majka, který používá stejná
jazyková data, ale kompletně proměnil jejich formát a~stejně tak
algoritmus, který nad nimi operuje -- tak bylo docíleno větší rychlosti
zpracování.~\parencite{majka}

\hypertarget{deriv}{%
\subsection{Deriv}\label{deriv}}

Třetím relevantním softwarovým řešením zabývající se slovní derivací je
projekt Masarykovy univerzity Deriv, jenž je víceúčelovým nástrojem pro
automatické zpracování přirozeného jazyka s~primárním cílem testovat
možnosti automatické slovotvorné analýzy. Jeho webového rozhraní se
skládá ze dvou základních funkcí -- vyhledávání podle formálního zadání
a~kategorizace vyhledaných dat. Samotný Deriv je založený na
automatickém morfologickém analyzátoru Ajka (později Majka), v~rámci
kterého využívá jeho morfologický slovník kmenů a~vyhledává v~něm
prostřednictvím morfologických značek a~regulárních
výrazu\footnote{Nástroj využívá regulární výrazy programovacího jazyka Perl verze 5.10 a~novější}.
Velkou výhodou aplikace je fakt, že jsou výsledky hledání propojeny
s~českými výkladovými slovníky (SSČ, SSJČ, PSJČ) a~s~některými českými
korpusy (konkrétně CzTenTen, SYN2000).~\parencite{deriv}

\hypertarget{derivancze}{%
\subsection{Derivancze}\label{derivancze}}

Prvním nástrojem, který zde uvádíme, je Reams and reams Něco hezkého
\parencite[516]{pala15}

\hypertarget{derivaux10dnuxed-suxedux165-derinet}{%
\section{Derivační síť
Derinet}\label{derivaux10dnuxed-suxedux165-derinet}}
