\hypertarget{uxfavod}{%
\chapter*{Úvod}\label{uvod}
\addcontentsline{toc}{chapter}{Úvod}}

Odvozování je v~českém jazyce považováno za nejčastější způsob vzniku
nových pojmenování. Na rozdíl od rodilých mluvčí existuje u~cizinců
studujících češtinu jako druhý jazyk potřeba znát jednotlivé odvozovací
morfémy, díky kterým by byli schopni podvědomě predikovat významy
neznámých slov. Tím by studující urychlili proces akvizice češtiny, ale
rovněž by docílili většímu pochopení studovaného jazyka jako takového.

Cílem této práce je navrhnout a~implementovat elektronický slovník
s~definicemi založenými na derivačních rysech slovotvorně motivovaných
slov, a~tak přispět k~větší edukaci slovotvorného systému českého
jazyka. Výsledný nástroj čerpá z~teoretických východisek tradiční
onomaziologické teorie slovotvorby Miloše Dokulila a~pracuje s~volně
přístupnými daty derivační sítě DeriNet. Za využití moderních hybridních
technologií pak tato data zpracovává do formy mobilní aplikace.

Práce samotná se tak skládá ze dvou částí -- teoretické a~praktické.
V~teoretické části je rozebrána problematika české slovotvorby -- jsou zde
představeny hlavní synchronní přístupy k~této lingvistické disciplíně
a~je zde popsáná onomaziologická teorie slovotvorby.

Praktická část se pak zabývá popisem výsledného nástroje, a~to nejprve
z~hlediska vytváření slovotvorných definic (spolu s~popisem zpracovaných
slovotvorných sufixů). Poté je nástroj představen z~technického pohledu,
tedy je zde rozebrán návrh a~implementace samotné mobilní aplikace.
