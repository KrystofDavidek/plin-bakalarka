\hypertarget{uxfavod}{%
\chapter*{Úvod}\label{uvod}
\addcontentsline{toc}{chapter}{Úvod}}

Derivace v~užším smyslu představuje v~českém jazyce, jakožto v~jazyce
s~vysoce rozvinutou flexí, nejčastější způsob utváření nových pojmenování.
Její podstata je nejčastěji založena na přidávání slovotvorného morfému
před (prefix) anebo za (sufix) slovní tvar. Rodilí mluvčí podvědomě
chápou sémantiku jednotlivých morfémů, a~proto jsou si schopni nejen
vyložit významy slov, které v~dané podobě nikdy neslyšeli, ale taktéž
takové novotvary vytvářet. U~cizinců učících se češtinu si je však
potřeba sémantiku těchto odvozovacích prostředků nejprve vědomě osvojit.
V~současné době neexistuje nástroj, který by byl určen pro účely
osvojování derivačních prostředků češtiny.

Cílem této práce je navrhnout a~implementovat elektronický slovník
s~definicemi založenými na derivačních rysech slovotvorně motivovaných
slov. Výsledná aplikace provádí pro zadaný vstup částečnou slovotvornou
analýzu, na základě které zadanému slovu přiřazuje definici vycházející
z~jeho struktury (potažmo strukturního významu).

Teoretickým východiskem pro vývoj aplikace je onomaziologické teorie
slovotvorby představená Milošem Dokulilem. Aplikace pracuje s~volně
přístupnými daty derivační sítě DeriNet, jež rovněž vychází
z~Dokulilovské teorie. Za využití moderních hybridních technologií pak
tato data zpracovává formou mobilní aplikace.

Práce se skládá ze dvou částí -- teoretické a~praktické. V~teoretické
části jsou nastíněny synchronní přístupy k~české slovotvorbě a~současně
je zde hlouběji popsáná onomaziologická teorie slovotvorby. Dále jsou
zde představeny již existující softwarové nástroje, které s~českou
slovotvorbou pracují.

Praktická část se soustřeďuje na popis výsledného nástroje, a~to nejprve
z~hlediska vytváření slovotvorných definic (spolu s~popisem zpracovaných
slovotvorných sufixů). V~neposlední řadě je nástroj představen
z~technické stránky, tedy je zde rozebrán návrh a~implementace mobilní
aplikace.
