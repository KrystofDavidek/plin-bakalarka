\hypertarget{uxfavod}{%
\chapter*{Úvod}\label{uvod}
\addcontentsline{toc}{chapter}{Úvod}}

U~cizinců učících se češtinu, jakožto jazyk s~vysoce rozvinutou flexí,
se dříve či později objevuje potřeba pochopit významy jednotlivých
odvozovacích prostředků. Tato potřeba pramení z~častého vytváření
novotvarů pomocí specifických jazykových prostředků, které pro takové
účely český jazyk má. Vědomá znalost významů jednotlivých odvozovacích
prostředků by mohla vést k~podstatně rychlejšímu procesu akvizice
češtiny jako druhého jazyka.

Cílem této práce je navrhnout a~implementovat elektronický slovník
s~definicemi založenými na derivačních rysech slovotvorně motivovaných
slov. Výsledná aplikace provádí pro zadaný vstup částečnou slovotvornou
analýzu, na základě které zadanému slovu přiřazuje definici vycházející
z~jeho struktury (potažmo strukturního významu).

Teoretickým východiskem pro vývoj aplikace je onomaziologické teorie
slovotvorby představená Milošem Dokulilem. Aplikace pracuje s~volně
přístupnými daty derivační sítě DeriNet, jež rovněž vychází
z~Dokulilovské teorie. Za využití moderních hybridních technologií pak
tato data zpracovává formou mobilní aplikace.

Práce se skládá ze dvou částí -- teoretické a~praktické. V~teoretické
části jsou nastíněny synchronní přístupy k~české slovotvorbě a~současně
je zde hlouběji popsáná onomaziologická teorie slovotvorby. Dále jsou
zde představeny již existující softwarové nástroje, které s~českou
slovotvorbou pracují.

Praktická část se soustřeďuje na popis výsledného nástroje, a~to nejprve
z~hlediska vytváření slovotvorných definic (spolu s~popisem zpracovaných
slovotvorných sufixů). V~neposlední řadě je nástroj představen
z~technické stránky, tedy je zde rozebrán návrh a~implementace mobilní
aplikace.
