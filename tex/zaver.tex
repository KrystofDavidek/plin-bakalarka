\hypertarget{zuxe1vux11br}{%
\chapter*{Závěr}\label{zaver}
\addcontentsline{toc}{chapter}{Závěr}}

Cílem této práce bylo navrhnout a~implementovat elektronický slovník
s~definicemi založenými na derivačních rysech slovotvorně motivovaných
slov. Aplikaci se podařilo implementovat dle zadaných požadavků, její
instalace a~testování proběhlo na mobilním zařízení s~operačním systémem
Android ve verzi 7.

Vzhledem ke složitosti slovotvorného systému českého jazyka jsou
v~aktuální verzi zpracována slova spadající do slovotvorných typů
\emph{-tel} a~\emph{-telka}. U~těchto slov byla navrhnuta lingvistická
pravidla, která slouží pro vytváření slovotvorných definic. S~ohledem na
cílovou skupinu (cizinci studující češtinu jako druhý jazyk) byl
vyhledán dvojjazyčný česko-anglický slovník, pomocí kterého jsou
slovotvorné definice kompletně lokalizovány v~anglickém jazyce.

V~budoucím vývoji aplikace se počítá s~navýšením počtu zpracovaných
slovotvorných typů (další v~pořadí je typ \emph{-ista}). Z~hlediska
uživatelského prostředí by se dále bylo vhodné zaměřit na implementaci
autentifikačního systému, v~rámci kterého by si tak uživatel mohl
ukládat již naučená slova (případně slovotvorné typy) do své vlastní
profilové stránky. Taktéž se do budoucna počítá s~propojením mobilní
aplikace s~webovou verzí, která bude svými vlastními funkcionalitami
předkládanou aplikaci doplňovat.

Cíle této práce byly splněny a~její výstupy by tak měly být rozšířeny
v~rámci případného navazujícího aplikovaného výzkumu.
